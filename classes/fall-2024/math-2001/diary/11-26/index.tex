\documentclass[12pt]{article}

% Packages
\usepackage[margin=1in]{geometry}
\usepackage{amsmath,amssymb,amsthm}
\usepackage{enumitem}
\usepackage{hyperref}
\usepackage{xcolor}
\usepackage{import}
\usepackage{xifthen}
\usepackage{pdfpages}
\usepackage{transparent}
\usepackage{listings}


\lstset{
    breaklines=true,         % Enable line wrapping
    breakatwhitespace=false, % Wrap lines even if there's no whitespace
    basicstyle=\ttfamily,    % Use monospaced font
    frame=single,            % Add a frame around the code
    columns=fullflexible,    % Better handling of variable-width fonts
}

\newcommand{\incfig}[1]{%
    \def\svgwidth{\columnwidth}
    \import{./Figures/}{#1.pdf_tex}
}
\theoremstyle{definition} % This style uses normal (non-italicized) text
\newtheorem{solution}{Solution}
\newtheorem*{proposition}{Proposition}
\newtheorem{problem}{Problem}
\newtheorem{lemma}{Lemma}
\theoremstyle{plain} % Restore the default style for other theorem environments

% Title information
\title{Proofs By Calculation}
\author{Jerich Lee}
\date{2024-11-26}

\begin{document}

\maketitle
Cool, so I am starting from the beginning beginning. Let's run it (LRI)—

\begin{problem}
   Let $a$ and $b$ be rational numbers and suppose that $a-b=4$ and $ab=1$. Show that $(a+b)^{2}=20$. 
\end{problem}
\begin{solution}
    \begin{proof}
        \begin{align}
            (a+b)^{2}&=(a-b)^{2}+4ab \\[10pt]
           &=4^{2}+4 \cdot 1 \\[10pt] 
           &= 20 
        \end{align}
    \end{proof}
\end{solution}
Cool, now how do we do this in Lean? Check this out:
\begin{lstlisting}
example {a b : \rat} (h1 : a - b = 4) (h2 : a * b = 1) : (a + b) ^ 2 = 20 :=
  calc
    (a + b) ^ 2 = (a - b) ^ 2 + 4 * (a * b) := by ring
    _ = 4 ^ 2 + 4 * 1 := by rw [h1, h2]
    _ = 20 := by ring
\end{lstlisting}
Ok let's do a new proof:
\begin{proof}
    \begin{align}
        d(af-be)&=(ad)f-dbe \\[10pt] 
        &=(bc)f-dbe \\[10pt] 
        &=b(cf)-dbe \\[10pt] 
        &=b(de)-dbe \\[10pt] 
        &=0
    \end{align}
\end{proof}
Woah. Let's do it in Lean!
\begin{lstlisting}
example {a b c d e f : \int} (h1 : a * d = b * c) (h2 : c * f = d * e) :
    d * (a * f - b * e) = 0 :=
  calc
    d * (a * f - b * e)
      = (a*d)*f-d*b*e := by ring
    _ = (b*c)*f-d*b*e := by rw [h1]
    _ = b*(c*f)-d*b*e := by ring
    _ = b*(d*e)-d*b*e := by rw [h2]
    _ = 0 := by ring
\end{lstlisting}
Fire. Onward. Yuh.
Last easy example: 
\begin{problem}
    
Let $a$ and $b$ be integers and suppose that $a=2b+5$ and $b=3$. Show that $a=11$.
\end{problem}
\begin{solution}
 \begin{proof}
    \begin{align}
        a &= 2b+5 \\[10pt] 
        &= 2 \cdot 3+ 5\\[10pt] 
        &= 11
    \end{align}
\end{proof}  
\end{solution}
Lean:
\begin{lstlisting}
    example {a b : \int} (h1: a = 2 * b + 5) (h2 : b = 3) : a = 11:=
    calc
      a = 2 * b + 5 := h1
      _ = 2 * 3 + 5 := by rw [h2]
      _ = 11 := by ring 
\end{lstlisting}
So I need to make sure to explicitly type in like \emph{slash}int and such or it won't work...noted.
\end{document}