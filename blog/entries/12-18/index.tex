\documentclass[12pt]{article}

% Packages
\usepackage[margin=1in]{geometry}
\usepackage{amsmath,amssymb,amsthm}
\usepackage{enumitem}
\usepackage{hyperref}
\usepackage{xcolor}
\usepackage{import}
\usepackage{xifthen}
\usepackage{pdfpages}
\usepackage{transparent}
\usepackage{listings}


\lstset{
    breaklines=true,         % Enable line wrapping
    breakatwhitespace=false, % Wrap lines even if there's no whitespace
    basicstyle=\ttfamily,    % Use monospaced font
    frame=single,            % Add a frame around the code
    columns=fullflexible,    % Better handling of variable-width fonts
}

\newcommand{\incfig}[1]{%
    \def\svgwidth{\columnwidth}
    \import{./Figures/}{#1.pdf_tex}
}
\theoremstyle{definition} % This style uses normal (non-italicized) text
\newtheorem{solution}{Solution}
\newtheorem*{proposition}{Proposition}
\newtheorem{problem}{Problem}
\newtheorem{lemma}{Lemma}
\theoremstyle{plain} % Restore the default style for other theorem environments
%

% Theorem-like environments
% Title information
\title{Almost done with finals}
\author{Jerich Lee}
\date{\today}

\begin{document}

\maketitle
I wanted to write this entry really quickly to just flow some thoughts and ideas as well as some reflections that I've had over the past couple days over finals.

I finished my MATH 447—Real Analysis final yesterday and boy it was tough. It took me all three hours to complete. I was stuck on this one question—\emph{Give an example of a function $f_{n}$ on $x \in [0,2]$ that is pointwise continuous but not uniformly continuous—prove both statements}. I was disappointed because it took me too long to figure that one out...I ended up choosing  
$f_{n}=n^{2}(\frac{x}{2})^{n}(2-x)$, but I couldn't figure out how to prove both statements in a clean manner.

Regardless, I am proud of how it went. I am expecting a B in the class, which is fine. The class not only taught me real analysis, but it taught me a lot of tools used in higher level math—i.e. it became a gateway for me to continue lifelong-learning in math, which is exactly why I took this class.

I am really grateful for my friends who are always there for me. A lot of my friends don't understand that I love learning. It is difficult to convey that I am working and learning, and this doesn't translate well when your friends are trying to make the most out of senior w.r.t. real life experiences, e.g. going out to the bars, having dinners together, having fun. It's definitely a sacrifice that I had to make, and sometimes I forget that I chose not to engage with their group activities and it makes me feel a bit regretful. 

It is not their fault. It is not my fault. I think people come and go, and that is just the nature of the universe. Things converge and diverge continuously. That is the way. The internet makes it harder for this process to happen I think though. Now that everything is stored and is visible to everybody, you are constructing a digital twin of yourself—that of which is presented to new people in which they can subconsciously models in their minds of you. This is of course an evolutionary trait, people want to make sure that the person you are is of good reputation. But this also makes it harder to continue to grow and develop intellectually and culturally I think—you end up locking yourself into what others \emph{perceive} you to be in their models—and when you do something that is true to your nature but doesn't quite fit in the models of others—you become a stranger. Or perhaps one could argue the opposite. I don't know. Like I said before, I'm just putting some thoughts out. 

I need to learn to give people space too. I create my own space, and then forget that other people do the same. Regardless, we will meet when the time is right. 

I also need to let go of some people. It's ok. I shouldn't feel like I am locked into a set of people that I used to click with, but don't anymore. Nothing is really set and stone. It's really fine, they won't worry about you and you shouldn't worry about them. 

Anyways, I have one more final exam today. This one I'm not too worried about, but it will definitely require a good production of brain power. Let's do it.
\end{document}
