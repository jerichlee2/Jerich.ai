\documentclass[12pt]{article}

% Packages
\usepackage[margin=1in]{geometry}
\usepackage{amsmath,amssymb,amsthm}
\usepackage{enumitem}
\usepackage{hyperref}
\usepackage{xcolor}
\renewcommand{\labelenumi}{(\alph{enumi})} % Change numbering to (a), (b), ...

% Define the solution environment with normal text
\theoremstyle{definition} % This style uses normal (non-italicized) text
\newtheorem{solution}{Solution}
\newtheorem*{proposition}{Proposition}
\newtheorem{problem}{Problem}
\newtheorem{lemma}{Lemma}
\theoremstyle{plain} % Restore the default style for other theorem environments
%


% Title information
\title{MATH 447: Real Variables - Homework \#10}
\author{Jerich Lee}
\date{\today}

\begin{document}

\maketitle
\begin{problem}[26.6]
    \noindent 
    Let $ s(x) = x - \frac{x^3}{3!} + \frac{x^5}{5!} - \cdots $ and $ c(x) = 1 - \frac{x^2}{2!} + \frac{x^4}{4!} - \cdots $ for $ x \in \mathbb{R} $.
    \begin{enumerate}
        \item Prove $ s' = c $ and $ c' = -s $.
        \item Prove $ (s^2 + c^2)' = 0 $.
        \item Prove $ s^2 + c^2 = 1 $.
    \end{enumerate}
    
    Actually, $ s(x) = \sin x $ and $ c(x) = \cos x $, but you do \textbf{not} need these facts.
\end{problem}
\begin{solution}
    \begin{enumerate}
        \item  \begin{enumerate}
            \item  \begin{proof}
        For $x\in\mathbb{\MakeUppercase{r}}$: 
   \begin{align}
    s(x) &= x-\frac{x^{3}}{3!}+\frac{x^{5}}{5!}-\ldots \\[10pt] 
    c(x) &= 1-\frac{x^{2}}{2!}+\frac{x^{4}}{4!}-\ldots \\[10pt] 
    \lim_{t \to x} \phi(t)&=\lim_{t \to x} \frac{f(t)-f(x)}{t-x}\\[10pt] 
    s(x)&=\sum_{n=0}^{\infty} \frac{x^{2n+1}(-1)^{n}}{(2n+1)}! \\[10pt] 
    &=\sum_{n=0}^{\infty} \frac{(-1)^{n}x^{2n+1}}{\left( 2n+1 \right)! } \\[10pt] 
    c(x) &= \sum_{n=0}^{\infty} \frac{(-1)^{n}x^{(2x)}}{\left( 2n \right)! }\\[10pt] 
    s^\prime (x)&=\sum_{n=0}^{\infty} \frac{(-1)^{n}(2n+1)x^{2n}}{\left( 2n+1 \right) !} \\[10pt] 
    &= \frac{(-1)^{n}x^{2n}}{\left( 2n \right) !} \\[10pt] 
    &= c(x)
   \end{align} 
    \end{proof}
    \item \begin{proof}
        \begin{align}
            c(x)&=\sum_{n=0}^{\infty} \frac{(-1)^{n}x^{2n}}{\left( 2n \right) !}\\[10pt] 
            c^\prime (x)&=\sum_{n=0}^{\infty} \frac{(-1)^{n}(2n)x^{2n-1}}{(2n)!}\\[10pt] 
            &= \frac{(-1)^{n}x^{2n-1}}{(2n-1)!}\\[10pt] 
            &= \sum_{n=1}^{\infty} \frac{(-1)^{n+1}x^{2n-1}}{(2n-1)!}\\[10pt] 
            &= -s(x)
        \end{align}
    \end{proof}
        \end{enumerate}    
        \item \begin{proof}
            \begin{align}
            \left( s^{2}+c^{2} \right)^\prime &=0 \\[10pt] 
            &= \sum_{k=0}^{\infty} \frac{(-1)^{k}x^{2k+1}}{(2k+1)!} + \sum_{k=0}^{\infty} \frac{(-1)^{k}x^{2k}}{(2k)!}\\[10pt] 
            &= \sum_{k=0}^{\infty} \left( \frac{(-1)^{k}x^{2k+1}}{(2k+1)!} + \frac{(-1)^{k}x^{2k}}{(2k)!} \right) \\[10pt] 
            &= 0
        \end{align}
        \end{proof}    
        \item todo
    \end{enumerate}
   \end{solution}
\begin{problem}[33.3]

    \noindent A function $ f $ on $[a, b]$ is called a \textit{step function} if there exists a partition 
    $$
    P = \{a = u_0 < u_1 < \cdots < u_m = b\}
    $$ 
    of $[a, b]$ —not $ P = \{a = u_0 < u_1 < \cdots < c_m = b\} $, as stated in the textbook— such that $ f $ is constant on each interval $(u_{j-1}, u_j)$, say $ f(x) = c_j $ for $ x $ in $(u_{j-1}, u_j)$.
    
    \begin{enumerate}
        \item Show that a step function $ f $ is integrable and evaluate $ \int_a^b f $.
    \end{enumerate}
    
\end{problem}
\begin{solution}
   \begin{proof}
    If $f$ is constant on every interval, then $M_i = m_i$. Then, $f$ is monotone and bounded on $\left( u_{j-1}, u_{j} \right) $, (in fact it is constant from $\left( u_{j-1},u_{j} \right) $), so therefore it is uniformly continuous, which implies that
    there exists a continuous extension to the closed set $[u_{j-1}, u_{j}] $. Invoking Theorem (3.38) from Ross (Piecewise Monotone), we show that $f\in \mathcal{\MakeUppercase{r}}$.   
   \end{proof} 
\end{solution}
\begin{problem}[33.7]
    Let $ f $ be a bounded function on $[a, b]$, so that there exists $ B > 0 $ such that $ |f(x)| \leq B $ for all $ x \in [a, b] $.

\begin{enumerate}
    \item Show
    $$
    U(f^2, P) - L(f^2, P) \leq 2B[U(f, P) - L(f, P)]
    $$
    for all partitions $ P $ of $[a, b]$. \textit{Hint:} $ f(x)^2 - f(y)^2 = [f(x) + f(y)] \cdot [f(x) - f(y)] $.
    \item Show that if $ f $ is integrable on $[a, b]$, then $ f^2 $ also is integrable on $[a, b]$.
\end{enumerate}
\end{problem}
\begin{solution}
    \begin{proof}
        $f$ is a bounded function on $[a,b]$, so there exists $B>0 \ s.t. \ \left\vert f(x) \right\vert \leq B$ for all $x\in[a,b]$.
        \begin{align}
            u(f^{2},p) &= \sum_{i=1}^{\infty} M_i^{2}\Delta x_{i} \\[10pt] 
            &= M_i^{2} = \sup f(x)^{2} \\[10pt] 
            &= \sum_{i=1}^{n} \left( M_i^{2}-m_i^{2} \right)\Delta x_i \\[10pt] 
            &= \sum_{i=1}^{n} \left( M_{i+m_i}  \right)\left( M_{i} - m_i \right)\Delta x_{i} \\[10pt] 
            &\leq \sum_{i=1}^{n} 2B\left( M_{i}-m_{i} \right)\Delta x_{i} \\[10pt] 
            &= 2B \sum_{i=1}^{\infty} \left( M_i - m_i \right)\Delta x_{i} \\[10pt] 
            &= 2B \left( U(f,p)-L(f,p) \right) 
        \end{align} 
    \end{proof}
\end{solution}
\begin{problem}[34.2]
    \noindent Calculate

\begin{enumerate}
    \item $ \lim_{h \to 0} \frac{1}{h} \int_3^{3+h} e^{t^2} \, dt $.
\end{enumerate}
\end{problem}
\begin{solution}
   \begin{proof}
    \begin{align}
        \int_{3}^{3+h}e^{t^{2}} \,\mathrm{d}t \\[10pt] 
        &= \lim_{h \to 0} \frac{F(3+h)-F(3)}{h}\\[10pt] 
    \end{align}
    By FTC I,
    \begin{align}
        F^\prime (3)  = e^{9} 
    \end{align}
   \end{proof} 
\end{solution}
\begin{problem}[34.5]
    \noindent  Let $ f $ be a continuous function on $ \mathbb{R} $ and define
$$
F(x) = \int_{x-1}^{x+1} f(t) \, dt \quad \text{for } x \in \mathbb{R}.
$$

Show $ F $ is differentiable on $ \mathbb{R} $ and compute $ F' $.
\end{problem}
\begin{solution}
    \begin{proof}
        \begin{align}
            F(x) &= \int_{x}^{0} f(t) \,\mathrm{d}t + \int_{0}^{x+1} f(t) \,\mathrm{d}t \\[10pt] 
            &= -\int_{0}^{x-1} f(t) \,\mathrm{d}t + \int_{0}^{x+1} f(t) \,\mathrm{d}t \\[10pt] 
        \end{align}
        By FTC II, we get:
        \begin{align}
            F^\prime (x_0) &= f(x_{0}+1) - f(x_{0}-1) 
        \end{align}
    \end{proof}
\end{solution}
\begin{problem}
    \noindent \textbf{A. [Bonus problem]} Suppose $ f $ is a continuous non-negative function on $[a, b]$, with 
$$
M = \max_{x \in [a, b]} f(x).
$$
For $ n \in \mathbb{N} $, let
$$
M_n = \left( \int_a^b f^n \, dt \right)^{1/n}.
$$
Prove that $ \lim M_n = M $.
\end{problem}
\begin{solution}
   \begin{proof}
    \begin{align}
        \left( \int_{a}^{b} f^{n}  \right)^{\frac{1}{n}} &\leq \left( (b-a)(M^{n}) \right)^{\frac{1}{n}} \\[10pt] 
        &= \lim_{n \to \infty} \underbrace{(b-a)^{\frac{1}{n}}}_{1} M = M 
    \end{align}
   \end{proof} 
\end{solution}
\end{document}
