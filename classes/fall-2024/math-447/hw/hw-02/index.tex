\documentclass[12pt]{article}

% Packages
\usepackage[margin=1in]{geometry}
\usepackage{amsmath,amssymb,amsthm}
\usepackage{enumitem}
\usepackage{hyperref}

% Define the solution environment with normal text
\theoremstyle{definition} % This style uses normal (non-italicized) text
\newtheorem{solution}{Solution}
\newtheorem*{proposition}{Proposition}
\newtheorem{problem}{Problem}
\newtheorem{lemma}{Lemma}
\theoremstyle{plain} % Restore the default style for other theorem environments
%

% Title information
\title{MATH 447: Real Variables - Homework \#2}
\author{Jerich Lee}
\date{\today}

\begin{document}

\maketitle

\begin{problem}[9.12]
    \begin{itemize}
        \item Assume all $s_n \neq 0$ and that the limit $L = \lim \left| \frac{s_{n+1}}{s_n} \right|$ exists.
        \begin{enumerate}
            \item[(a)] Show that if $L < 1$, then $\lim s_n = 0$. \textit{Hint:} Select $a$ so that $L < a < 1$ and obtain $N$ so that $|s_{n+1}| < a|s_n|$ for $n \ge N$. Then show $|s_n| < a^{n-N}|s_N|$ for $n > N$.
            \item[(b)] Show that if $L > 1$, then $\lim |s_n| = +\infty$. \textit{Hint:} Apply (a) to the sequence $t_n = \frac{1}{|s_n|}$; see Theorem 9.10.
        \end{enumerate}
    \end{itemize}
\end{problem}

\begin{solution}
    \begin{enumerate}
        \item \begin{proof}
        $\forall s_n, s_n \neq 0, \lim_{n \to \infty} \left| \frac{s_{n+1}}{s_n} \right| = L.$
    
        \begin{proposition}
             If $L < 1$, then $\lim_{n \to \infty} s_n = 0.$
        \end{proposition}
    
        $\forall \epsilon > 0, \exists N \in \mathbb{N} \text{ s.t. } n > N \implies \left| \frac{s_{n+1}}{s_n} - L \right| < \epsilon.$
        This implies that $\frac{s_{n+1}}{s_n} < \epsilon + L.$ Then we select a value $a$ such that $L < a < 1.$
        \begin{align}
            s_{n+1} &< (\epsilon + L) s_n \tag{1} \\
            a &= \epsilon + L \tag{2}
        \end{align}
    
        To show that $|s_n| < a^{n-N} |s_N|$, we will show the following:
        \begin{align}
            |s_n| &= \left| \frac{s_n}{s_{n-1}} \cdot \frac{s_{n-1}}{s_{n-2}} \cdot \dots \cdot \frac{s_{N+1}}{s_N} \cdot |s_N| \right| \tag{4} \\
            |s_n| &< a^{n-N} |s_N| \tag{5}
        \end{align}
    
        $\frac{1}{a^N} |s_N|$ is a constant, so we can rename it as the following:
        \begin{align}
            C = \frac{1}{a^N} |s_N| \tag{6}
        \end{align}
    
        Then, we can say the following:
        \begin{align}
            |s_n| < a^n \tag{7}
        \end{align}
    
        If $a < 1$, we know that $\lim_{n \to \infty} a^n = 0.$ By Theorem 9.7 (b) in Ross, we know that $\lim_{n \to \infty} a^n = 0.$
        Therefore, by Theorem 9.2 in Ross, $\lim_{n \to \infty} |s_n| = 0.$ $\qed$
    \end{proof}
\item    \begin{proposition}
         If $L > 1$, then $\lim_{n \to \infty} |s_n| = \infty.$
    \end{proposition}
    
    \begin{proof}
        Let $t_n = \frac{1}{|s_n|}$. Our goal is to prove $\lim_{n \to \infty} t_n = 0.$ Then:
        \begin{align}
            L &= \lim_{n \to \infty} \left| \frac{s_{n+1}}{s_n} \right| \notag \\
            &= \lim_{n \to \infty} \left| \frac{t_n}{t_{n+1}} \right| \tag{9}
        \end{align}
    
        $\forall \epsilon > 0, \exists N \in \mathbb{N} \text{ s.t. } n > N \implies \left| \frac{t_n}{t_{n+1}} - L \right| < \epsilon.$
        \begin{align}
            \frac{t_n}{t_{n+1}} &< \left( \frac{\epsilon + L}{a} \right) t_n \tag{11}
        \end{align}
    
        Applying the same process as in (a), we can say that $L < a < 1.$ Then:
        \begin{align}
            t_{n+1} &< at_n \tag{12} \\
            \frac{t_{n+1}}{t_n} &< a \tag{13} \\
            |t_n| &= \left| \frac{t_n}{t_{n-1}} \cdot \frac{t_{n-1}}{t_{n-2}} \cdot \dots \cdot \frac{t_{N+1}}{t_N} \cdot |t_N| \right| \tag{14} \\
            C &= a^{-N} |t_N| \tag{15}
        \end{align}
    
        We can say that $|t_n| < a^n C$ if $a < 1$, so $\lim_{n \to \infty} a^n = 0.$ Then $\lim_{n \to \infty} t_n = 0.$ Therefore:
        \begin{align}
            \lim_{n \to \infty} \frac{1}{|s_n|} = 0 \tag{16}
        \end{align}
        By Theorem 9.10 in Ross, $\lim_{n \to \infty} |s_n| = \infty.$ $\qed$
    \end{proof}

    \end{enumerate}
    \end{solution}

\begin{problem}[9.14]
    Let $p > 0$. Use Exercise 9.12 to show
    \begin{align}\lim_{n \to \infty} \frac{a^n}{n^p} = 
    \begin{cases} 
    0 & \text{if } |a| \leq 1 \\ 
    +\infty & \text{if } a > 1 \\ 
    \text{does not exist} & \text{if } a < -1.
    \end{cases}\end{align}
    Hint: For the $a > 1$ case, use Exercise 9.12(b).

\end{problem}
$\frac{1}{n^{2}}$ 
$\sum_{n=1}^{\infty} \frac{1}{n^{2}}$ 
\begin{solution}
\begin{enumerate}
    \item\begin{proof}
        We begin with $s_n = \lim_{n \to \infty} \frac{a^n}{n^P}, |a| \leq 1$. Applying Problem 1:
        \begin{align}
            \left| \frac{s_{n+1}}{s_n} \right| &= \frac{a^{n+1}}{(n+1)^P} \cdot \frac{n^P}{a^n} \tag{17} \\
            &= a \left( \frac{n^P}{(n+1)^P} \right) \tag{18} \\
            &= a \tag{19} \quad \text{(since } \lim_{n \to \infty} \frac{n^P}{(n+1)^P} = 1\text{)}.
        \end{align}
    
        The $\lim_{n \to \infty} \left| \frac{s_{n+1}}{s_n} \right|$ exists, so Problem 1 applies. According to the result from Problem 1a, if $\lim_{n \to \infty} \left| \frac{s_{n+1}}{s_n} \right| < 1$, then $\lim_{n \to \infty} s_n = 0$. When $|a| \leq 1$, $\lim_{n \to \infty} \frac{a^n}{n^P} = a$. Therefore, its limit must be 0 when $|a| \leq 1$, as desired. $\qed$
    \end{proof} 
    \item \begin{proof}
        We begin with $s_n = \lim_{n \to \infty} \frac{a^n}{n^P}, a > 1$. Applying Problem 1:
        \begin{align}
            \left| \frac{s_{n+1}}{s_n} \right| &= \frac{a^{n+1}}{(n+1)^P} \cdot \frac{n^P}{a^n} \tag{20} \\
            &= a \left( \frac{n^P}{(n+1)^P} \right) \tag{21} \\
            &= a \tag{22} \quad \text{(since } \lim_{n \to \infty} \frac{n^P}{(n+1)^P} = 1\text{)}.
        \end{align}
    
        The $\lim_{n \to \infty} \left| \frac{s_{n+1}}{s_n} \right|$ exists, so Problem 1 applies. According to the result from Problem 1a, if $\lim_{n \to \infty} \left| \frac{s_{n+1}}{s_n} \right| > 1$, then $\lim_{n \to \infty} s_n = +\infty$. When $a > 1$, $\lim_{n \to \infty} \frac{a^n}{n^P} = a$. Therefore, its limit must be $+\infty$ when $a > 1$, as desired. $\qed$
    \end{proof}
    \item \begin{proof}
        We begin with $s_n = \lim_{n \to \infty} \frac{a^n}{n^P}, a < -1$. To show that $\lim_{n \to \infty} \frac{a^n}{n^P} = \text{DNE}$, we will show that there exists more than one limit for the sequence $s_n$. Let $s_{n_1}$ be the subsequence such that $n$ is even. Let $s_{n_2}$ be the subsequence such that $n$ is not even. 
    
        To show that a limit is divergent, the following must be satisfied: 
        \begin{align}
        \forall M > 0, \exists N \in \mathbb{N} \text{ s.t. } n \geq N \implies \left( \frac{a^{2n}}{(2n)^P} \right) > M \tag{23}
        \end{align}
    
        Choose $N$ such that $N > \frac{P \ln 2 + \ln M}{2 (\ln a - P)}$. Then, for all $n > N$, this implies:
        \begin{align}
            n(2 \ln a - P) &> P \ln 2 + \ln M \tag{24} \\
            2n \ln a - P \ln n &> \ln (2^P M) \tag{25} \\
            \ln \left( \frac{a^{2n}}{N^P} \right) &> \ln (2^P M) \tag{26} \\
            \frac{a^{2n}}{(2n)^P} &> M \tag{27}
        \end{align}
    
        This implies that the $\lim_{n \to \infty} s_{n_1}$ is divergent. The same reasoning follows with $\lim_{n \to \infty} s_{n_2}$. We then get:
        \begin{align}
            \lim_{n \to \infty} s_{n_1} &= \lim_{n \to \infty} \frac{a^{2n}}{(2n)^P} \tag{28} \\
            \lim_{n \to \infty} s_{n_2} &= \lim_{n \to \infty} \frac{a^{2n+1}}{(2n+1)^P} \tag{29}
        \end{align}
    
        When $a < -1, n \in \mathbb{N}$, $a^{2n}$ is always positive and $a^{2n+1}$ is always negative. We then get:
        \begin{align}
            \lim_{n \to \infty} s_{n_1} &= \lim_{n \to \infty} \frac{a^{2n}}{(2n)^P} = +\infty \tag{31} \\
            \lim_{n \to \infty} s_{n_2} &= \lim_{n \to \infty} \frac{a^{2n+1}}{(2n+1)^P} = -\infty \tag{32}
        \end{align}
    
        There are two subsequences of $s_n$ with two distinct limits, so by Theorem 11.8 iii) in Ross, the limit of $s_n$ with $a < -1$ does not exist. $\qed$
    \end{proof}
\end{enumerate}
\end{solution}

% Add more problems and solutions as needed


\begin{problem}[10.6]
    \begin{itemize}
        \item[(a)] Let $(s_n)$ be a sequence such that
        \begin{align}
        |s_{n+1} - s_n| < 2^{-n} \quad \text{for all } n \in \mathbb{N}.
        \end{align}
        Prove $(s_n)$ is a Cauchy sequence and hence a convergent sequence.
        \item[(b)] Is the result in (a) true if we only assume $|s_{n+1} - s_n| < \frac{1}{n}$ for all $n \in \mathbb{N}$?
    \end{itemize}


\end{problem}

\begin{solution}
    \begin{enumerate}
        \item \begin{proof}
            Let $s_n$ be a sequence such that $\forall n \in \mathbb{N}, |s_{n+1} - s_n| < 2^{-n}$. To show that a sequence is Cauchy, we must satisfy the following: 
            \begin{align}
            \forall \epsilon > 0, \exists N \in \mathbb{N} \text{ s.t. } n, m > N \implies |s_n - s_m| < \epsilon.
            \end{align}
            We will show this by showing that:
            \begin{align}
            \forall \epsilon, \exists N_0 \in \mathbb{N} \text{ s.t. } n > N_0 \implies |2^{-n} - 0| < \epsilon.
            \end{align}
        
            Then we can bound $s_n < \epsilon$ for all $n, m \in \mathbb{N}$, thereby showing $s_n$ is Cauchy. We will solve for $N$ in the expression:
            \begin{align}
                \frac{1}{2^N} &< \epsilon \tag{34} \\
                2^N &> \frac{1}{\epsilon} \tag{35} \\
                N &> \log_2 \left( \frac{1}{\epsilon} \right) \tag{36}
            \end{align}
        
            Then, we can choose $N = \log_2 \left( \frac{1}{\epsilon} \right)$. Then, $\forall n > N$,
            \begin{align}
                |s_{n+1} - s_n| &< 2^{-n} \tag{37} \\
                |s_m - s_n| &< 2^{-n} < \epsilon \tag{38} \\
                |s_m - s_n| &< \epsilon \tag{39}
            \end{align}
        
            Therefore, the sequence $s_n$ is Cauchy, as required. $\qed$
        \end{proof}
        \item \begin{proof}
            We can follow a similar line of reasoning from the previous question. We want $\forall n \in \mathbb{N}, s_n = |s_{n+1} - s_n| < \frac{1}{n}$. So we will prove that:
            \begin{align}
            \forall \epsilon > 0, \exists N \text{ s.t. } m, n > N \implies |s_{n+1} - s_n| < \epsilon.
            \end{align}
        
            \begin{align}
                |s_{n+1} - s_n| &< \frac{1}{n} \tag{40} \\
                \left| \frac{1}{n} - 0 \right| &< \epsilon \tag{41}
            \end{align}
        
            To determine $N$, we will use algebra as follows:
            \begin{align}
                \frac{1}{N} &< \epsilon \tag{42} \\
                \frac{1}{\epsilon} &< N \tag{43}
            \end{align}
        
            We will choose $N = \frac{1}{\epsilon}$. Then, $n > N$ implies:
            \begin{align}
                \frac{1}{n} &< \epsilon \tag{44}
            \end{align}
        
            And hence,
            \begin{align}
                \left| \frac{1}{n} - 0 \right| &< \epsilon \tag{45} \\
                |s_m - s_n| &< \frac{1}{n} < \epsilon \tag{46} \\
                |s_m - s_n| &< \epsilon \tag{47}
            \end{align}
        
            Therefore, the sequence satisfies the Cauchy criterion. $\qed$
        \end{proof}
    \end{enumerate}
\end{solution}

\begin{problem}[10.8]
    Let $(s_n)$ be an increasing sequence of positive numbers and define $\sigma_n = \frac{1}{n} (s_1 + s_2 + \cdots + s_n)$. Prove $(\sigma_n)$ is an increasing sequence.
\end{problem}

\begin{solution}
    \begin{proof}
        Suppose by contradiction, that 
        \begin{align}
        \frac{1}{n+1}(s_1 + s_2 + \cdots + s_n + s_{n+1}) < \frac{1}{n}(s_1 + s_2 + \cdots + s_n).
        \end{align}
        Then,
        \begin{align}
            \sum_{i=1}^{n+1} s_i &< \frac{n+1}{n} \sum_{i=1}^n s_i \tag{48} \\
            \sum_{i=1}^{n+1} s_i &< \left( 1 + \frac{1}{n} \right) \sum_{i=1}^n s_i \tag{49} \\
            s_{n+1} &< \frac{1}{n} \sum_{i=1}^n s_i \tag{50} \\
            ns_{n+1} &\leq \sum_{i=1}^n s_n \tag{51} \\
            s_{n+1} &< ns_n \tag{52}
        \end{align}
    
        Contradiction.
    \end{proof}
\end{solution}

\begin{problem}[10.10]
    Let $s_1 = 1$ and $s_{n+1} = \frac{1}{3}(s_n + 1)$ for $n \geq 1$. 
    \begin{enumerate}[label=(\alph*)]
        \item Find $s_2$, $s_3$ and $s_4$.
        \item Use induction to show $s_n > \frac{1}{2}$ for all $n$.
        \item Show $(s_n)$ is a decreasing sequence.
        \item Show $\lim s_n$ exists and find $\lim s_n$.
    \end{enumerate}
\end{problem}

\begin{solution}
    \begin{enumerate}
        \item Let $s_1 = 1$ and $s_{n+1} = \frac{1}{3}(s_n + 1)$ for $n \geq 1$. Then:
        \begin{enumerate}
            \item $s_1 = 1$
            \item $s_2 = \frac{1}{3}(s_1 + 1) = \frac{1}{3}(2) = \frac{2}{3}$
            \item $s_3 = \frac{1}{3} \left( \frac{2}{3} + 1 \right) = \frac{5}{9}$
            \item $s_4 = \frac{1}{3} \left( \frac{5}{9} + 1 \right) = \frac{14}{27}$
        \end{enumerate}
        \item \begin{proof}
            We will use induction. The base case is as follows:
            \begin{align}
            s_1 = 1 \tag{53}
            \end{align}
        
            The induction hypothesis is:
            \begin{align}
            \forall n \geq 1, \frac{1}{2} < s_{n+1} < s_n < 1 \tag{54}
            \end{align}
        
            The inductive step is as follows:
            \begin{align}
                \frac{1}{3}(s_{n+1} + 1) &< s_{n+1} \tag{55} \\
                \frac{s_{n+1}}{3} + \frac{1}{3} &< s_{n+1} \tag{56} \\
                \frac{1}{3} &< \frac{2s_{n+1}}{3} \tag{57} \\
                \frac{1}{2} &< s_{n+1} \tag{58}
            \end{align}
        
            To finish the proof, we need $\frac{1}{3}(s_{n+1} + 1) > \frac{1}{2}$:
            \begin{align}
                \frac{s_{n+1}}{3} - \frac{1}{6} &> 0 \tag{59} \\
                \frac{1}{6}(2s_{n+1} - 1) &> 0 \tag{60} \\
                s_{n+1} &> \frac{1}{2} \tag{61}
            \end{align}
        \end{proof}
        \item \begin{proof}
            To prove that $\forall n \geq 1, s_1 = 1, s_n = \frac{1}{3}(s_n + 1)$ is decreasing, we will show that $\forall n \geq 1, s_{n+1} \leq s_n$:
            \begin{align}
                \frac{1}{3}(s_{n+1} + 1) &< s_n \tag{62} \\
                \frac{s_n}{3} + \frac{1}{3} &< s_n \tag{63} \\
                \frac{1}{3} &< \frac{2s_n}{3} \tag{64} \\
                \frac{1}{2} &< s_n \tag{65}
            \end{align}
        
            The last line was proved by induction in part (b) of this problem. So $s_n$ is decreasing. $\qed$
        \end{proof}
        \item \begin{proof}
            To show that $\lim_{n \to \infty} s_n$ exists, we can state that because $s_n$ is decreasing, it is monotone. Because $s_n > \frac{1}{2}$ for all $n \in \mathbb{N}$, the sequence is also bounded. Therefore, by Theorem 10.2 in Ross, $s_n$ converges and must have a limit.
        
            Let $\epsilon > 0$, $S = \{s_n : n \in \mathbb{N}\}$, and $u = \inf s_n$. $u + \epsilon$ is not a lower bound of $S$, so $\exists N \text{ s.t. } s_N < u + \epsilon$ for all $n \geq N$.
        
            Thus, $u \leq s_n < u + \epsilon \implies |s_n - u| < \epsilon$. From part (b), we proved $\frac{1}{2} < s_{n+1} < s_n < 1$. So $u = \inf s_n = \frac{1}{2}$. Therefore, by Theorem 10.2 from Ross, 
            \begin{align}
            \lim_{n \to \infty} s_n = u = \frac{1}{2}.
            \end{align}
            $\qed$
        \end{proof}
    \end{enumerate}
\end{solution}

\begin{problem}[10.12]
    Let $t_1 = 1$ and $t_{n+1} = \left[1 - \frac{1}{(n+1)^2}\right] \cdot t_n$ for $n \geq 1$.
    \begin{enumerate}[label=(\alph*)]
        \item Show $\lim t_n$ exists.
        \item What do you think $\lim t_n$ is?
        \item Use induction to show $t_n = \frac{n+1}{2n}$.
        \item Repeat part (b).
    \end{enumerate}
\end{problem}

\begin{solution}
    \begin{enumerate}
        \item \begin{proof}
            To show that $t_n$ is decreasing, we will show that $t_{n+1} < t_n$ for all $n \in \mathbb{N} \text{ s.t. } n \geq 1$:
            \begin{align}
                \left[1 - \frac{1}{(n+1)^2}\right] \cdot t_n &< t_n \tag{66} \\
                \left[1 - \frac{1}{(n+1)^2}\right] &< 1 \tag{67} \\
                -\frac{1}{(n+1)^2} &< 0 \tag{68} \\
                0 &< \frac{1}{(n+1)^2} \tag{69}
            \end{align}
        
            The last line is always true for $n \in \mathbb{N}$, so the $\lim_{n \to \infty} t_n$ exists. We have shown that $t_n$ is monotone, and therefore has a limit. $\qed$
        \end{proof}
        \item I think the limit is $\frac{1}{2}$.
        \item \begin{proof}
            We will use induction. For the base case, $n = 1$:
            \begin{align}
                t_n &= \frac{n + 1}{2n} \tag{70} \\
                t_1 &= \frac{2}{2} = 1 \tag{71}
            \end{align}
        
            The inductive hypothesis is as follows:
            \begin{align}
            t_n = \frac{n + 1}{2n} \quad \text{for } n \geq 1 \tag{72}
            \end{align}
        
            The inductive step:
            \begin{align}
                t_{n+1} &= \frac{(n+1) + 1}{2(n+1)} \tag{74} \\
                &= \frac{n + 2}{2(n+1)} \tag{75} \\
                &= \left[ 1 - \frac{1}{(n+1)^2} \right] t_n \tag{76} \\
                &= \left[ 1 - \frac{1}{(n+1)^2} \right] \left( \frac{n+1}{2n} \right) \tag{77} \\
                &= \frac{(n+1)^2 (n+1)}{2n(n+1)^2} - \frac{n+1}{2n(n+1)^2} \tag{78} \\
                &= \frac{n+2}{2(n+1)} \tag{79}
            \end{align}
        
            Thus, the proof is complete.
        \end{proof}
        \item \begin{proof}
            \begin{align}
                \lim_{n \to \infty} \frac{n + 1}{2n} &= \lim_{n \to \infty} \frac{1 + 1/n}{2} \tag{80} \\
                &= \frac{1}{2} \tag{81}
            \end{align}
        
            Equation (80) is established from Theorem 9.10 in Ross. $\qed$
        \end{proof} 
    \end{enumerate}
\end{solution}


\end{document}