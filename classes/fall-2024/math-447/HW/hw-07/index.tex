\documentclass[12pt]{article}

% Packages
\usepackage[margin=1in]{geometry}
\usepackage{amsmath,amssymb,amsthm}
\usepackage{enumitem}
\usepackage{hyperref}
\usepackage{xcolor}

% Define the solution environment with normal text
\theoremstyle{definition} % This style uses normal (non-italicized) text
\newtheorem{solution}{Solution}
\newtheorem*{proposition}{Proposition}
\newtheorem{problem}{Problem}
\newtheorem{lemma}{Lemma}
\theoremstyle{plain} % Restore the default style for other theorem environments
%


% Title information
\title{MATH 447: Real Variables - Homework \#7}
\author{Jerich Lee}
\date{\today}

\begin{document}

\maketitle

\begin{problem}[22.3]
Prove that if $E$ is a connected subset of a metric space $(S, d)$, then its closure $E^-$ is also connected.
\end{problem}
\begin{solution}
    \begin{proof}
        To solve this problem, we will prove the contrapositive, i.e., if \(E^{-}\) is disconnected, then \(E\) is also disconnected. Because \(E^{-}\) is disconnected, there exist \(U_1, U_2\) such that:
    \begin{enumerate}
        \item \(E^{-}\subseteq U_1 \cup U_2\) 
        \item \((E^{-}\cap U_1)\cap (E^{-}\cap U_2)=\emptyset\)
        \item \((E^{-}\cap U_1) \neq \emptyset, (E^{-1}\cap U_{2})\neq \emptyset\)   
    \end{enumerate}  By the definition of closure, \(E \subset E^{-}\). Then, 
    \begin{align}
        (E\cap U_1)\cap (E\cap U_2)\subset (E^{-}\cap U_1)\cap (E^{-}\cap U_2) \\[10pt] 
        (E\cap U_1)\cap (E\cap U_2)=\emptyset 
    \end{align}  Therefore, the following holds true:
    \begin{enumerate}
    \item \(E \subset E^{-} \subseteq U_1 \cup U_2\implies E \subseteq U_1 \cup U_2\)  
    \item \((E\cap U_1)\cap (E\cap U_2)=\emptyset\)
    \item \((E\cap U_1)\neq \emptyset, (E\cap U_{2})\neq \emptyset\)   
    \end{enumerate}
    Therefore, \(E\) is also disconnected.
\end{proof}
\end{solution}
\begin{problem}
Prove that an intersection of convex sets in $\mathbb{R}^n$ is convex.
\end{problem}
\begin{solution}
    \begin{proof}
        Suppose we have sets \(E,F\) that are both convex. We wish to find \(E\cup F\) is convex. We know the following:
        \begin{align}
            \forall x, y \in E, 0<t<1 \implies tx+(1-t)y\in E\\[10pt] 
            \forall u, v \in F, 0<t<1 \implies tu+(1-t)v \in F
        \end{align}
        Choose \(\forall a,b \in E\cup F\). Then,
        \begin{align}
            ta+(1-t)b\in E\\[10pt] 
            ta+(1-t)b \in F
        \end{align} 
        By the definition of an intersection of a set, we know that \(x\in P\) iff \(x\in E_{\alpha}\) for every \(\alpha \in A\). Let \(a \in E\), and \(b \in F\), and \(E = E_1\), \(F = E_2\), \(E_1, E_2 \in E_{\alpha}\). \(P=E\cap F\). By the above, we know that \(ta+(1-t)b\in E\cap F\). Therefore, the intersection of convex sets \(E\) and \(F\) is also convex.        
    \end{proof}
\end{solution}
\begin{problem}
On the metric space $\mathbb{R}^n$ (with the Euclidean metric $d$), denote by $P_i$ $(1 \leq i \leq n)$ the projection onto the $i$-th coordinate. Specifically, $P_i : \mathbb{R}^n \to \mathbb{R}$ takes $\vec{x} = (x_1, \dots, x_n)$ to $x_i$. Prove that $P_i$ is Lipschitz.
\end{problem}
\begin{solution}
    \begin{proof}
        The idea is to choose the largest value of \(P_i(s)\), where \(s\in \mathbb{\MakeUppercase{r}}^{n}\), and \(P_i\) is the projection function of \(\mathbb{\MakeUppercase{r}}^{n}\), i.e., choose the projection with the largest magnitude. By the triangle inequality, we know that the difference between the largest projection of values \(s, t \in \mathbb{\MakeUppercase{r}}^{n}\) is always less than or equal to their respective Euclidean distances. As a result, we can always bound the differences between the difference of the outputs by the difference of the inputs by some constant value \(k\), thereby making the function \(P_i\) Lipschitz. For elements \( s, t \in \mathbb{R}^{n} \), choose \(k=1\). Then,
        \begin{align}
            \left\vert s_i - t_i \right\vert < (1)d(s,t) \\[10pt] 
            =\left(  \sum_{i=1}^{n} \left\vert s_i - t_i \right\vert^{2}  \right)^{\frac{1}{2}}
        \end{align}
        Therefore, \(P_i\) is Lipschitz. 
         \end{proof}
\end{solution}
\begin{problem}
Denote by $\ell_1$ the set of all absolutely convergent series: the elements of $\ell_1$ are sequences $a = (a_i)_{i=1}^{\infty}$ with $\sum_{i=1}^{\infty} |a_i| < \infty$. For $a = (a_i)_{i=1}^{\infty}$ and $b = (b_i)_{i=1}^{\infty}$, define 
\[
d(a,b) = \sum_{i=1}^{\infty} |a_i - b_i|.
\]
\begin{enumerate}
    \item Prove that $d$ is a metric.
    \item Prove that the function $f : \ell_1 \to \mathbb{R}: (a_i) \mapsto \sum_{i=1}^{\infty} a_i$ is Lipschitz.
    \item Determine whether the function $g: \mathbb{R} \to \ell_1$, taking $t \in \mathbb{R}$ to the sequence $\left( \frac{t^2}{2^i} \right)_{i=1}^{\infty}$, is uniformly continuous.
\end{enumerate}
\end{problem}
\begin{solution}
        \begin{enumerate}
            \item \item To show that \(d\) is a metric, we need to show that \(d\) satisfies the three criteria of a metric space:
            \begin{enumerate}
                \item Non-degeneracy:
                \begin{align}
                  d(x,y)=0 \iff x=y \\[10pt] 
                  d(a,b) = \sum_{i=1}^{\infty} \left\vert a_i - b_{i} \right\vert  
                \end{align} 
                \begin{proof}
                    \(\implies:\) 
                    \begin{align}
                        \sum_{i=1}^{\infty} \left\vert a_{i}-b_{i} \right\vert = \sum_{i=1}^{\infty} \left\vert 0 \right\vert =0 
                    \end{align} 
                    \(\impliedby:\) 
                    \begin{align}
                        \sum_{i=1}^{\infty} \left\vert a_{i}-b_{i} \right\vert =0 \implies \left\vert a_{i}-b_{i} \right\vert =0 \\[10pt] 
                        a_{i}=b_{i}=0
                    \end{align}
                \end{proof}
                \item Symmetry:
                \begin{align}
                    \forall x,y \in \ell_1, d(x,y)=d(y,x) 
                \end{align}
                \begin{proof}
                    \begin{align}
                        \left\vert a_{i}-b_{i} \right\vert =\left\vert b_{i}-a_{i} \right\vert 
                    \end{align}
                \end{proof}
                \item Triangle Inequality: 
                \begin{align}
                    \forall x,y,z \in \ell_1, d(x,y)+d(y,z)\geq d(x,z)
                \end{align} 
                    \begin{proof}
                    \begin{align}
                        \sum_{i=1}^{\infty} \left\vert a_i -b_{i} \right\vert +\sum_{i=1}^{\infty} \left\vert b_{i}-c_{i}  \right\vert \geq \sum_{i=1}^{\infty} \left\vert a_i -c_{i} \right\vert \\[10pt] 
                        \sum_{i=1}^{\infty} a_i = A, \sum_{i=1}^{\infty} b_i = B, \sum_{i=1}^{\infty} c_i = C \label{hw-5}\\[10pt] 
                        \left\vert A-B \right\vert +\left\vert B-C \right\vert \geq \left\vert A -C \right\vert \\[10pt] 
                        \left\vert A-B \right\vert +\left\vert B-C \right\vert \geq \left\vert (A-B)+(B-C) \right\vert \label{ti}
                    \end{align} 
    Where \autoref{hw-5} is proven in Homework 5 Problem 3.1, and \autoref{ti} is proven by the triangle inequality.
                \end{proof}
            \end{enumerate}
            \item \begin{align}
                f: \ell_1 \to \mathbb{\MakeUppercase{r}} : (a_{i}) \mapsto \sum_{i=1}^{\infty} a_{i}\\[10pt] 
                s,t \in \ell_1, d^{*}(f(s), f(t))\leq kd(s,t)
            \end{align}
            \begin{proof}
             We wish to find:
            \begin{align}
                \sum_{i=1}^{\infty} s_{i} - \sum_{i=1}^{\infty} t_{i} \leq k \sum_{i=1}^{\infty} \left\vert a_{i}-b_{i} \right\vert \\[10pt] 
                \left\vert S \right\vert -\left\vert T \right\vert \leq k\left\vert S-T \right\vert 
            \end{align} 
            Choose \(k=1\). Then:
            \begin{align}
                \left\vert \left( S-T \right)+T  \right\vert \leq \left\vert S-T \right\vert +\left\vert T \right\vert \\[10pt] 
                \left\vert S \right\vert \leq \left\vert S-T \right\vert +\left\vert T \right\vert 
            \end{align}
            By the triangle inequality established in Problem 4.1, at \(k = 1\), \(f\) is Lipschitz. 
            \end{proof}
            \item To prove that \(g: \mathbb{\MakeUppercase{r}} \to \ell_1\) is uniformly continuous, we will use the definition, i.e., 
            \begin{align}
                \forall \varepsilon>0, \exists \delta \ s.t. \ x,y\in S \text{ and } \left\vert x-y \right\vert <\delta \implies \left\vert f(x)-f(y) \right\vert <\varepsilon
            \end{align} 
            We will determine \(\delta\) by the following discussion:
            \begin{align}
                f(x)-f(y) = \left(\frac{x^{2}}{2^{n}}- \frac{y^{2}}{2^{n}}   \right) =\\[10pt] 
                \left( \frac{x^{2}-y^{2}}{2^{n}} \right) = \left( \frac{(x+y)(x-y)}{2^{n}} \right)  \\[10pt] 
                \left( \frac{(x+y)(x-y)}{2^{n}} \right) <\varepsilon \\[10pt] 
                (x-y) < \frac{\varepsilon \cdot 2^{n}}{(x+y)} \\[10pt] 
            \end{align} 
            Choose \(\delta=\frac{\varepsilon \cdot 2^{n}}{(x+y)}\). Then, \(\left\vert x-y \right\vert <\delta \) implies:
            \begin{align}
                \left\vert f(x)-f(y) \right\vert = \left\vert \left(\frac{x^{2}}{2^{n}}- \frac{y^{2}}{2^{n}}   \right) \right\vert  =\\[10pt] 
                \left( \frac{x^{2}-y^{2}}{2^{n}} \right) = \left( \frac{(x+y)(x-y)}{2^{n}} \right) < \left( \frac{(x+y)\delta}{2^{n}} \right) \\[10pt] 
                \frac{(x+y)\cdot \varepsilon \cdot 2^{n}}{(x+y)\cdot 2^{n}} = \varepsilon \\[10pt]  
            \end{align} 
            Therefore, 
            \begin{align}
                \left\vert x-y \right\vert < \delta \implies  \left\vert f(x)-f(y) \right\vert < \varepsilon
            \end{align}
            Then \(g\) is uniformly continuous.
        \end{enumerate}
        

\end{solution}
\end{document}
