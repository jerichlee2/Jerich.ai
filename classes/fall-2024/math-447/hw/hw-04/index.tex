\documentclass[12pt]{article}

% Packages
\usepackage[margin=1in]{geometry}
\usepackage{amsmath,amssymb,amsthm}
\usepackage{enumitem}
\usepackage{hyperref}

% Theorem-like environments


% Define the solution environment with normal text
\theoremstyle{definition} % This style uses normal (non-italicized) text
\newtheorem{solution}{Solution}
\newtheorem*{proposition}{Proposition}
\newtheorem{problem}{Problem}
\newtheorem{lemma}{Lemma}
\theoremstyle{plain} % Restore the default style for other theorem environments
%

% Title information
\title{MATH 447: Real Variables - Homework \#4}
\author{Jerich Lee}
\date{\today}

\begin{document}

\maketitle

\begin{problem}[13.3(a)]
Let \(B\) be the set of all bounded sequences \(\mathbf{x} = (x_{1}, x_{2}, \ldots)\) and define \(d(\mathbf{x}, \mathbf{y})=\sup \left\{ \left\vert x_{j}-y_{j}  \right\vert :j=1,2, \ldots   \right\} \).
\begin{enumerate}
    \item Show \(d\) is a metric for \(B\). 
\end{enumerate}
\end{problem}

\begin{solution}
We will show that \(d\) is a metric for \(B\) by showing that it satisfies the symmetry, non-degeneracy, and triangle inequality with respect to \(B\), i.e. the set of all bounded sequences.
\begin{proof}
 \begin{enumerate}
    \item Symmetry: 
    We wish to show that \(d(\mathbf{x}, \mathbf{y})=d(\mathbf{y}, \mathbf{x})\).  \begin{align} d(\mathbf{x}, \mathbf{y}) = \sup\left\{ \left\vert x_{j}-y_{j} \right\vert : j=1,2, \ldots   \right\} \\[10pt]  d(\mathbf{x}, \mathbf{y}) = \sup\left\{ \left\vert y_{j}-x_{j} \right\vert : j=1,2, \ldots   \right\} \\[10pt] \end{align} From the properties of absolute value, we know that \(\left\vert a-b \right\vert = \left\vert b-a \right\vert \). Setting \(a=x_{j}\) and \(b=y_{j} \), we can state: \begin{align} \left\vert x_{j} -y_{j}  \right\vert =\left\vert y_{j} -x_{j}  \right\vert \end{align} \item Non-degeneracy We wish to show that \(d(\mathbf{x},\mathbf{y} )=\sup \left\{ \left\vert x_{j}-y_{j}  \right\vert : j=1,2, \ldots   \right\}=0 \) iff \(\mathbf{x} =\mathbf{y} \).
   \begin{align}
    \left\vert x_{j} -y_{j}  \right\vert =0  \iff x_{j} =y_{j} \\[10pt] 
    x_{j} -y_{j} =0    
   \end{align}
   \item Triangle Inequality
   We wish to show that \(d(\mathbf{x}, \mathbf{z})\leq d(\mathbf{x},\mathbf{y})+d(\mathbf{y}, \mathbf{z})\).
   \begin{align}
\text{RHS} = d(\mathbf{x}, \mathbf{z})+d(\mathbf{y}, \mathbf{zy})\\[10pt] 
= \sup\left\{ \left\vert x_{j}-y_{j} \right\vert : j=1,2, \ldots   \right\}+ \sup\left\{ \left\vert y_{j}-z_{j} \right\vert : j=1,2, \ldots   \right\} \\[10pt] 
\geq  \sup\left\{ \left\vert x_{j}-y_{j} \right\vert +\left\vert y_{j} -z_{j}  \right\vert : j=1,2, \ldots   \right\}\\[10pt] 
\geq \sup \left\{ (x_{j} -y_{j} )+(y_{j} -z_{j} ) \right\} \\[10pt] 
=\sup \left\{ \left\vert x_{j} -z_{j} \right\vert   \right\} =\text{LHS} 
   \end{align}
\end{enumerate} 
Lines 8 and 9 can be shown in Homework 3 Problem 4.
\end{proof}
\end{solution}

\begin{problem}[13.4]
    Prove (iii) and (iv) in Discussion 13.7
    \begin{enumerate}
        \item The union of \emph{any} collection of open sets is open.
        \item The intersection of \emph{finitely many} open sets is again an open set.
    \end{enumerate}
\end{problem}

\begin{solution}
    \item \begin{proof}
        Suppose we have a collection of open sets, \(\left\{ U_k\right\} \). Then if \(x\in \left\{ U_k\right\} \), then \(x\in \cup_k \left\{ U_k \right\} \). Because \(\left\{ U_k \right\} \) is a collection of open sets, \(\cup_k \left\{ U_k \right\} \) is also itself an open set. 
    \end{proof}
    \item \begin{proof}
        Suppose we have a collection of finitely open sets, \(\left\{ U_k \right\}\). Then, for each element \(x\) in the intersection \(\cap_k U_k\), there exists a neighborhood \(N_k\) about the element \(x\) for each open set \(U_{ k=1,2, \ldots  k}  \) such that \(N_k\subset U_k, k=1,2, \ldots  k \). For each \(N_k\), there exists a radius \(r_k\). We can choose the smallest radius out of all neighborhoods \(N_k\) such that \(r_{\mathop{\min}}=\mathop{\min} \left\{ N_1,N_{2}, \ldots  N_k   \right\} \). Then, this neighborhood is a subset of all open sets \(U_k\), \(N_{k_{\mathop{\min}(r) } }\subset \left\{U_k \right\} \).  Therefore, the intersection \(\cap_k U_k\) is open for a finite collection of open sets.     
    \end{proof}
\end{solution}

\begin{problem}[13.10(a)]
Show that the interior of each of the following sets is the empty set.
\begin{enumerate}
    \item \(\left\{ \frac{1}{n}:n\in\mathbb{\MakeUppercase{n}}  \right\} \) 
\end{enumerate}
\end{problem}

\begin{solution}
   \item \begin{proof}
    The definition of the interior point of a set \(E\) is a point that contains at least one neighborhood such that \(N\subset E\). The interior of a set \(E\) is a set that contains only interior points. A set \(E\) is called \emph{open} if the set \(E\) is equal to its interior. Therefore, to show that the interior of the set \(E=\left\{ \frac{1}{n}:n\in \mathbb{\MakeUppercase{n}}  \right\} \) is the empty set, we must show that there exists a point in \(E\) such that its neighborhood
    contains elements that are included in the set \(\left\{ \frac{1}{n}:n\in\mathbb{\MakeUppercase{n}}  \right\}\). 
    We will show that for any point in \(E\), there is no neighborhood of \(s_{o} \) such that \(B_{r}^{o}(s_{o} )\subset E \) for all \(r\). We will first show that between any two rational numbers, there exists an irrational number.
    \begin{align}
        0<\frac{1}{\sqrt{2} }<1 \\[10pt] 
        r_{1} + 0 < r_{1} + \frac{1}{\sqrt{2} }(r_{1} -r_{2} )<r_{1} +(r_{2} -r_{1} )\\[10pt] 
        r_{1} <r_{1} +\frac{1}{\sqrt{2} }(r_{2} -r_{1} )<r_{2} \\[10pt] 
    \end{align}
    Suppose that the radius of any neighborhood about any point in \(E\) is a rational number. Then, by above, there exists members of \(B_{r}^{o}(s_{o} )\) that are irrational, and \(\notin E \), as \(E\) contains only rational numbers. Now suppose that we set our radius \(r\) to be an irrational number. We know that between any rational and irrational number exists an irrational number. Let \(A\)  be a rational number, and \(C\)  to be an irrational number. Then, \(B=\frac{A+C}{2}\) is irrational, and \(A<B<C\). Therefore, we have shown that there is no value of radius \(r\) that satisfies \(B_{r}^{o}(s_{o} )\subset E\) for any rational point \(s_{o} \) in \(E\).  e
   \end{proof} 
\end{solution}

\begin{problem}[13.12(b)]   
Let \((S,d)\) be any metric space.
\begin{enumerate}
    \item Show that the finite union of compact sets in \(S\) is compact.
\end{enumerate}
\end{problem}

\begin{solution}
   \begin{proof}
    To show that the finite union of compact sets in \(S\) is compact, we will use the property of compactness of each set \(U_k\) in the finite union \(\cup_k \left\{  U_k\right\} \). 
    If a set \(U_k\) is compact, then there exists a finite subcover for every open cover of \(U_k\). Then, there exists \(\cup_k U_k\) for finitely many \(k\).
   \end{proof} 
\end{solution}

\begin{problem}[A]
    Show that a sequence in a metric space \((S,d)\) cannot have more than one limit.
\end{problem}

\begin{solution}
   We will show that if a sequence \(s_{n} \) converges to \(L_{1} \) and \(L_{2} \), \(\forall \varepsilon >0\), \(\exists N_{1}\ s.t. \ m,n >N \implies \left\vert s_{n} -L_{1}   \right\vert <\frac{\varepsilon}{2}\) and \(\exists N_{2} \ s.t. \ n>N_{2} \implies \left\vert s_{n} -L_{2}  \right\vert <\frac{\varepsilon}{2}\). Choose \(N_{\text{max} }=\text{max}(N_{1},N_{2} )  \). Then, \(n>N_{\text{max} } \implies \)
   \begin{align}
    d\left( s_{n}  -L_{1}  + (L_{2}-s_{n}) \right)  <  d\left( \frac{\varepsilon}{2} \right)  +  d\left( \frac{\varepsilon}{2} \right)< \varepsilon \\[10pt] 
   =  d\left( \frac{\varepsilon}{2} \right) <\varepsilon 
   \end{align}
   This implies that \(L_{1} =L_{2} \), so there cannot be more than one limit in \((S,d)\). 
\end{solution}

\begin{problem}[B]
    \begin{enumerate}
        \item Suppose a sequence \(s_{n} \) in a metric space \((S,d)\) converges to \(s\in S\). Prove that any subsequence of \(s_{n} \) converges to \(s\) as well.
    \end{enumerate}
\end{problem}

\begin{solution}
    Let \(s_{n} \) be a sequence in \(S\). By the given, \(\forall \varepsilon >0,\exists N\ s.t. \ n>N\implies d(s_{n}-s)\). Let \(s_{n_{k}} \) be a subsequence of \(s_{n} \). We know that \(n_{k} >k\) for all \(k\). 
    Now let \(s=\lim_{n \to \infty} s_{n} \) and let \(\varepsilon >0\). There exists \(N\) such that \(n>N\) implies \(d(s_{n} -s)<\varepsilon \). Now \(k>N\) implies \(n_{k} >N\), implying \(d(s_{n_{k} }-s )<\varepsilon \). Therefore,
    \begin{align}
        \lim_{k \to \infty} s_{n_{k} }=s 
    \end{align}   
\end{solution}
\begin{problem}[C]
    Suppose \((S,d)\) is a complete metric space, and \(E \subset S\). We can view \(E\) as a metric space, equipped with the metric inherited from \(S\). Prove that \(E\) is complete iff it is a closed subset of \(S\). 
\end{problem}

\begin{solution}
    \(\implies :\) Suppose that \(E\) is not complete. Then, \(E\) does not contain its limit points. By definition, a closed set is a set that contains all of its limit points. Therefore, \(E\) cannot be closed.
    \(\impliedby :\) Suppose that \(E\) is not a closed subset of \(S\). Then, \(E\) does not contain all of its limit points. Suppose that \(E\) is complete. Then, every Cauchy sequence in \(E\) converges to some limit point in \(E\).
\end{solution}
\begin{problem}[D]
    Suppose \(s_{n} \) is a Cauchy sequence in a metric space \((S,d)\) which has a convergent subsequence. Is it true that the sequence \(s_{n} \) itself converges?
\end{problem}

\begin{solution}
   We know that the following holds:
   \begin{align}
    \forall \varepsilon >0, \exists N\ s.t. \ m,n>N\implies d(s_{m} -s_{n} )<\varepsilon 
   \end{align} 
   Want: If \(s_{n} \) has a convergent subsequence, then \(L\in S\), \(d(s_{n_{k} }-L )<\varepsilon\implies d(s_{n} -L)<\varepsilon  \).
   Choose \(N\) such that \(n>N\implies d(s_{n_{k}  } -L)<\frac{\varepsilon}{2}\).
   Choose \(N_{1} \) such that \(d(s_{n_{k+1} }-s_{n_{k} }  )<\frac{\varepsilon}{2}\)  
   Then we get:
   \begin{align}
    d(s_{n_{k+1} }-s_{n_{k} }+s_{n_{k} }-L   )<\frac{\varepsilon}{2}+\frac{\varepsilon}{2}<\varepsilon \\[10pt] 
    d(s_{n_{k+1} }-L )<\varepsilon 
   \end{align}
\end{solution}

\begin{problem}[Bonus Problem]
    Is the metric space \((B,d)\) defined in Problem 13.3(a) complete?
\end{problem}

\begin{solution}
    \begin{proposition}
        If \(B\) is the set of all bounded sequences \(\mathbf{x} = (x_{1}, x_{2}, \ldots, x_n )\), endowed with the distance function \(d(\mathbf{x}, \mathbf{y})=\sup \left\{ \left\vert x_i - y_{i} \right\vert : i=1,2, \ldots n \right\}\), then the metric space \((B,d)\) is complete.
    \end{proposition}
    \begin{proof}

        We wish to find a proof that \((B, d)\) such that \(B\) is the set of all bounded sequences, and \(d\) is the distance defined as:
        \begin{align}
            d(\mathbf{x}, \mathbf{y})=\sup \left\{ \left\vert x_i - y_{i} \right\vert : i=1,2, \ldots n \right\} 
        \end{align} 
        The idea is to treat the set \(B\) of sequences as a set of infinite dimensional vectors in \(\mathbb{\MakeUppercase{R}} \). 
        We will utilize and prove the following lemma: 
        \begin{lemma}
                Consider a sequence \((\mathbf{x}^{k})_k\) in \(\mathbb{\MakeUppercase{r}}^n\), with \(\mathbf{x}^{k}=(x_i^{k})^{n}_{i=1}\)   
            \begin{enumerate}
                \item \((\textbf{x}^{(k)})_{k}\) is Cauchy iff \((\textbf{x}^{k})_{k}\)  is Cauchy for \(1 \leq i \leq n \)
                \item \((\textbf{x}^{(k)})_{k}\) converges to \(\mathbf{x} = (x_{i})^{n}_{i=1}\) iff \(\lim_{k \to \infty} x_{i}^{(k)}\) for \(1 \leq i \leq n \) 
            \end{enumerate} 
        \end{lemma}
        \noindent
       \(\implies :\)  To prove part one of the lemma above, we will suppose that \((\textbf{x}^{k})_{k}\) is Cauchy. Fix \(i\). Want: \((x_{i}^{(k)})_{k}\) is Cauchy. For \(\varepsilon > 0\) we need to find \(N\ s.t. \ \left\vert x_{i}^{(k)} - x_{i}^{(m)} \right\vert <\varepsilon \) for \(k, m>N\). We need to find an \(N\ s.t. \ d(\mathbf{x}^{k}, \mathbf{x}^{m})<\varepsilon\) for all \(k,m>N\). \(\varepsilon > d(\mathbf{x}^{(k)}, \mathbf{x}^{(m)})=\sup \left\{ \left\vert  x_{i}- y_{i}\right\vert : i= 1,2, \ldots , n   \right\} \geq \left\vert x_{i}-y_{i} \right\vert \). Therefore, this \(N\) works in the forward direction.
        \(\impliedby :\) Suppose \((x_{i}^{(k)})_{k}\) is Cauchy \(\forall \ i\). We want to find: \((\mathbf{x}^{(k)})_{k}\) is Cauchy. We fix \(\varepsilon >0\). We need to find \(N\ s.t. \ d(\mathbf{x}^{k}, \mathbf{x}^{m})<\varepsilon \) for \(k, m>N\). For \(1\leq i\leq n\) find \(N_{i} \in \mathbb{\MakeUppercase{n}} \ s.t. \ \left\vert x_{i}^{(k)}- x_{i}^{(m)} \right\vert < \varepsilon\) for \(k,m>N_{i} \). Then, we know that:
        \begin{align}
            x_{i}^{(k)} <\varepsilon + x_{i}^{(m)} \\[10pt] 
            x_{i}^{(m)} < \varepsilon + x_{i}^{(k)} 
        \end{align}
        Therefore, by the properties of the least upper bound, we have the following:
        \begin{align}
          \sup x_{i}^{(k)} \leq \varepsilon + x_{i}^{(m)}\\[10pt] 
          \sup x_{i}^{(m)} \leq \varepsilon + x_{i}^{(k)}\\[10pt] 
        \end{align}
        We can prove a quick fact proposition about differences of supremums of sequences:
        \begin{proposition}
       \[ 
          \sup (s_{n} -t_{n} )\leq \sup s_{n} - t_{n} \\[10pt]  
       \] 
        \end{proposition}
        \begin{proof}
            \begin{align}
                s_{n} \leq \sup s_{n} \\[10pt] 
                t_{n} \leq \sup t_{n} \\[10pt] 
                s_{n} - t_{n} \leq \sup s_{n} -\sup t_{n} \\[10pt] 
                \sup (s_{n} -t_{n} ) \leq \sup s_{n} -\sup t_{n} 
            \end{align}
        \end{proof}
        Therefore, we can state the following:
        \begin{align}
            \sup (x_{i}^{(k)} -x_{i}^{(m)})\leq \sup x_{i}^{(k)}-\sup x_{i}^{(m)}\\[10pt] 
            \leq (x_{i}^{(m)} +\varepsilon )-(x_{i}^{(k)} +\varepsilon )
        \end{align}
        We know from the implication that the following is true:
        \begin{align} 
            \left\vert x_{i}^{(k)}- x_{i}^{(m)} \right\vert< \varepsilon \\[10pt] 
            \sup (x_{i}^{(k)}-x_{i}^{(m)}    )<\varepsilon 
        \end{align}
        Therefore, we can say the following:
        \begin{align} 
            \sup \left(  \left\vert x_{i}^{(k)}-x_{i}^{(m)} \right\vert\right)      <\varepsilon 
        \end{align}
        To show the second part of Lemma 1, we can use similar reasoning as the first part of Lemma 1 to show that the limit exists \(B\).  
    \end{proof}
\end{solution}

\end{document}