\documentclass[12pt]{article}

% Packages
\usepackage[margin=1in]{geometry}
\usepackage{amsmath,amssymb,amsthm}
\usepackage{enumitem}
\usepackage{hyperref}
\usepackage{xcolor}
\usepackage{import}
\usepackage{xifthen}
\usepackage{pdfpages}
\usepackage{transparent}
\usepackage{listings}


\lstset{
    breaklines=true,         % Enable line wrapping
    breakatwhitespace=false, % Wrap lines even if there's no whitespace
    basicstyle=\ttfamily,    % Use monospaced font
    frame=single,            % Add a frame around the code
    columns=fullflexible,    % Better handling of variable-width fonts
}

\newcommand{\incfig}[1]{%
    \def\svgwidth{\columnwidth}
    \import{./Figures/}{#1.pdf_tex}
}
\theoremstyle{definition} % This style uses normal (non-italicized) text
\newtheorem{solution}{Solution}
\newtheorem*{proposition}{Proposition}
\newtheorem{problem}{Problem}
\newtheorem{lemma}{Lemma}
\newtheorem{theorem}{Theorem}
\theoremstyle{plain} % Restore the default style for other theorem environments
%

% Theorem-like environments
% Title information
\title{HW 1—MATH E-222: Abstract Algebra I}
\author{Jerich Lee}
\date{\today}

\begin{document}

\maketitle
\noindent
Read $\S 1.1$ and pages $38-42$ in Artin.  
\begin{problem}[1.1.7]
    Find a formula for $\begin{bmatrix}
        1 & 1 &  1 \\
        0 & 1 &  1 \\
        0 & 0 &  1 \\
    \end{bmatrix}^{n}$, and prove it by induction. 
\end{problem}
\begin{solution}
    
\end{solution}
\begin{problem}[1.1.16]
A square matrix $A$ is called \emph{nilpotent} if $A^{k}=0$ for some $k>0$. Prove that if $A$ is nilpotent, then $I+A$ is invertible.   
\end{problem}
\begin{solution}
    
\end{solution}
\begin{problem}[1.1.17]
   \begin{enumerate}
    \item Find infinitely many matrices $B$ such that $BA=I_2$ when 
    \begin{align}
        A= \begin{bmatrix}
            2 &  3 \\
            1 &  2 \\
            2 &  5 \\
        \end{bmatrix}
    \end{align}
    \item Prove that there is no matrix $C$ such that $AC=I_3$. 
   \end{enumerate} 
\end{problem}
\begin{solution}
    
\end{solution}
\end{document}
