\documentclass[12pt]{article}

% Packages
\usepackage[margin=1in]{geometry}
\usepackage{amsmath,amssymb,amsthm}
\usepackage{enumitem}
\usepackage{hyperref}
\usepackage{xcolor}
\usepackage{import}
\usepackage{xifthen}
\usepackage{pdfpages}
\usepackage{transparent}
\usepackage{listings}

\lstset{
    breaklines=true,         % Enable line wrapping
    breakatwhitespace=false, % Wrap lines even if there's no whitespace
    basicstyle=\ttfamily,    % Use monospaced font
    frame=single,            % Add a frame around the code
    columns=fullflexible,    % Better handling of variable-width fonts
}

\newcommand{\incfig}[1]{%
    \def\svgwidth{\columnwidth}
    \import{./Figures/}{#1.pdf_tex}
}
\theoremstyle{definition} % Normal (non-italicized) text
\newtheorem{solution}{Solution}
\newtheorem{proposition}{Proposition}
\newtheorem{problem}{Problem}
\newtheorem{lemma}{Lemma}
\newtheorem{theorem}{Theorem}
\newtheorem{remark}{Remark}
\newtheorem{note}{Note}
\theoremstyle{plain} % Restore the default style for other theorem environments

% Title information
\title{Congruence mod n; (Z/nZ)*}
\author{Jerich Lee}
\date{\today}

\begin{document}
\maketitle
Do exercises $2.9.2, 2.9.4, 2.9.5, 2.9.8$
Read $\S 2.10$  
\begin{problem}[2.9.2]
   \noindent
   \begin{enumerate}
    \item Prove that the square $a^{2}$ of an integer $a$ is congruent to $0$ or $1$ modulo $4$.
    \item What are the possible values of $a^{2}$ modulo $8$?    
   \end{enumerate} 
\end{problem}
\begin{solution}
    
\end{solution}
\begin{problem}[2.9.4]
   Prove that every integer $a$ is congruent to the sum of its decimal digits modulo $9$.  
\end{problem}
\begin{solution}
    
\end{solution}
\begin{problem}[2.9.5]
   Solve the congruence $2x \equiv 5$:
   \noindent
   \begin{enumerate}
    \item modulo $9$
    \item modulo $6$  
   \end{enumerate}  
\end{problem}
\begin{solution}
    
\end{solution}
\begin{problem}[2.9.8]
   Use Proposition (2.6) to prove the Chinese Remainder Theorem: Let $m,n,a,b$ be integers, and assume that the greatest common divisor of $m$ and $n$ is $1$. Then there is an integer $x$ such that $x \equiv a$ (modulo $m$) and $x \equiv b$(modulo $n$).
   \begin{proposition}[2.6]
    Let $a,b$ be integers, not both zero, and let $d$ be the positive integer which generates the subgroup $a\mathbb{{Z}}+b\mathbb{{Z}}$. Then
    \noindent
    \begin{enumerate}
        \item $d$ can be written in the form $d=ar + bs$ for some integers $r$ and $s$ .
        \item $d$ divides $a$ and $b$ .
        \item If an integer $e$ divides $a$ and $b$, it also divides $d$ .
    \end{enumerate} 
   \end{proposition}     
\end{problem}
\begin{solution}
    
\end{solution}
\end{document}
