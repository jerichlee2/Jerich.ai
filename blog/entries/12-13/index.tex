\documentclass[12pt]{article}

% Packages
\usepackage[margin=1in]{geometry}
\usepackage{amsmath,amssymb,amsthm}
\usepackage{enumitem}
\usepackage{hyperref}
\usepackage{xcolor}
\usepackage{import}
\usepackage{xifthen}
\usepackage{pdfpages}
\usepackage{transparent}
\usepackage{listings}


\lstset{
    breaklines=true,         % Enable line wrapping
    breakatwhitespace=false, % Wrap lines even if there's no whitespace
    basicstyle=\ttfamily,    % Use monospaced font
    frame=single,            % Add a frame around the code
    columns=fullflexible,    % Better handling of variable-width fonts
}

\newcommand{\incfig}[1]{%
    \def\svgwidth{\columnwidth}
    \import{./Figures/}{#1.pdf_tex}
}
\theoremstyle{definition} % This style uses normal (non-italicized) text
\newtheorem{solution}{Solution}
\newtheorem*{proposition}{Proposition}
\newtheorem{problem}{Problem}
\newtheorem{lemma}{Lemma}
\theoremstyle{plain} % Restore the default style for other theorem environments
%

% Theorem-like environments
% Title information
\title{Why is social media so difficult?}
\author{Jerich Lee}
\date{December 13, 2024}

\begin{document}

\maketitle
I was thinking last night about how hard it is to be on social media these days. There's so much junk on the internet—reels, memes, sometimes even completely random things–it's like I'm constantly wading in a pool of trash while the posts made by people that I want to see are trying their best
to float atop. As a of right now, I have 6 social media accounts...all of them which I value very much. It seems like each one of them is targeted at a slice of my personality:
\begin{enumerate}
    \item IG: where most of my friends live, mainly sports and memes (this can get so dull so fast tbh.)
    \item linkedin: also where most of my friends live, but most are incentivized for career stuff.
    \item mastodon: My safe place. This is like where I can be around people that are passionate, creative, and entertaining. It's also where I can be my complete self. Really grateful for the mathstodon community.
    \item youtube: I love youtube, but I also don't like it. So much entertainment, but also so much junk(less than IG). 
    \item github: not really a social media so it's deemed good. I'm really appreciative of github, open source code is some of the greatest things to have come out of the digital age.
    \item stack exchange: i'm not too active here, but it always feels good to be surrounded by people who are working on hard problems.
\end{enumerate}
I have a difficult time balancing my time with IG and Youtube the most. I always have this strange sense of guilt whenever I catch myself doomscrolling...the algorithms are too good. I am also aware that my friends are in the same boat, but they don't feel that guilt as much as I do.
If it were up to me, I would get rid of IG. I got rid of Snapchat a year ago, and I always think about how burdensome it was to constantly be responding to friends. But once you get rid of such a platform, you realize how different you are from everyone...it's almost as if the entire population is transitioning to this form of instant-tech normalcy. 

Knowing this, I am stuck in this pickle. Marcus Aurelius in \emph{Meditations} said that it was important to be social—you are proportionally defined by the connections to others. Douglas Hofstadter argued that the entire concept of \emph{consciousness} is built upon your connections with others in \emph{I am a Strange Loop}. But what happens when the people around you become slowly seduced to the powers of mathematical algorithms—to the point where everybody, including your mentors, are converging in social phase space? You become a \emph{stranger}, an outsider. 

As these thoughts ruminate, I think about how times were so different just a couple decades ago. People sent physical mail, you had to walk into the toy store and actually buy the toy via human-to-human interaction. There really is something so special about talking with a human face-to-face at the restaurant or at the desert shop. It really does ache when I walk into a coffee shop on Green Street and the man who is working the cashier points me to the standing tablet with the same template software as every other food place on Green. But I also think about how much more difficult other aspects of life were, such as travel. It's so nice to be able to just order an Uber to get to places. 

I have fought with these thoughts for some time, but my conclusion is that this is a natural order of the universe and instead of fighting against mathematical algorithms, I need to be accepting. I love humanity so much, and it is a journey that I must follow with my peers. But it is important for me to be conscious and active of communities where I can really feel comfortable, e.g. mathstodon. 

My hope is that the people who are in control of these mathematical algorithms will do good and not evil, and in the meantime I will do my best to do good on these platforms. I have much hope for humanity, as I believe that a moderate use of social media is also one of the best things to have ever happened in the digital age. When done right, is a place where people can share their uniqueness and be free. 

My fear is that people will become paralyzed on platforms due to judgement. The more social media becomes more prevalent, the more information is known about every person. This information is \emph{forever} cemented in the silicon sphere...it's like getting a permanent tattoo on your body for everyone to see. If things were so permanent, how do we know what to even put on our bodies? This encourages people to display \emph{perfection}...but in a universe devoid of such perfection, what does this even entail?

\end{document}
