\documentclass[12pt]{article}

% Packages
\usepackage[margin=1in]{geometry}
\usepackage{amsmath,amssymb,amsthm}
\usepackage{enumitem}
\usepackage{hyperref}
\usepackage{xcolor}
\usepackage{import}
\usepackage{xifthen}
\usepackage{pdfpages}
\usepackage{transparent}




\newcommand{\incfig}[1]{%
    \def\svgwidth{\columnwidth}
    \import{./Figures/}{#1.pdf_tex}
}
\theoremstyle{definition} % This style uses normal (non-italicized) text
\newtheorem{solution}{Solution}
\newtheorem*{proposition}{Proposition}
\newtheorem{problem}{Problem}
\newtheorem{lemma}{Lemma}
\theoremstyle{plain} % Restore the default style for other theorem environments
%

% Theorem-like environments
% Title information
\title{Learning LEAN!}
\author{Jerich Lee}
\date{\today}

\begin{document}

\maketitle
I thought it would be a cool idea to learn LEAN—a functional programming language used for proof-assistance—sometime this semester, but I realized that there was a large
learning curve to it, as there were so many freakin files and setups and such and things just got a bit annoying. Now that I have some time on my hands this break, I thought it would be a good idea to finish the course, as I already understand the material—I just need to get comfortable
with the syntax.
\\
My ultimate goal is to use LEAN for my math classes next semester. That would be so cool! It will be difficult though. Maybe as an exercise, I will go through all of Real Analysis using LEAN and publish it on my website. I will write daily logs about my learning progress and we will take it from there. \begin{verbatim}¯\(°_o)/¯\end{verbatim} 
\end{document}
