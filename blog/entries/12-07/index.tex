\documentclass[12pt]{article}

% Packages
\usepackage[margin=1in]{geometry}
\usepackage{amsmath,amssymb,amsthm}
\usepackage{enumitem}
\usepackage{hyperref}
\usepackage{xcolor}
\usepackage{import}
\usepackage{xifthen}
\usepackage{pdfpages}
\usepackage{transparent}
\usepackage{listings}


\lstset{
    breaklines=true,         % Enable line wrapping
    breakatwhitespace=false, % Wrap lines even if there's no whitespace
    basicstyle=\ttfamily,    % Use monospaced font
    frame=single,            % Add a frame around the code
    columns=fullflexible,    % Better handling of variable-width fonts
}

\newcommand{\incfig}[1]{%
    \def\svgwidth{\columnwidth}
    \import{./Figures/}{#1.pdf_tex}
}
\theoremstyle{definition} % This style uses normal (non-italicized) text
\newtheorem{solution}{Solution}
\newtheorem*{proposition}{Proposition}
\newtheorem{problem}{Problem}
\newtheorem{lemma}{Lemma}
\theoremstyle{plain} % Restore the default style for other theorem environments
%

% Theorem-like environments
% Title information
\title{Flying Through "Thing" Space}
\author{Jerich Lee}
\date{December 07, 2024}

\begin{document}

\maketitle

I've been thinking a lot about how cool/crazy it is to be able to make abstractions—\emph{mappings}—to things that pop up frequently. There's this idea
called \emph{chunking} that I've noticed in a lot of things that I do on a daily basis—e.g., Rubik's cubing, guitar, mathematics. It's cool that once you 
are comfortable with a finite set of "things", you can combine these "things" in sequences to make new "things", transform these "things" (inversion, rotation, translation). 

In mathematics, one would call these \emph{linearly independent basis vectors}, and neat results are just elements of the span of this space.
In Rubik's cubing, you start out learning the \emph{sexy move}. Once you know the sexy move, you learn the inverse, then the lefty variants of it. It turns out that the \emph{sexy move} is just an element of the class known as "triggers". Once one learns these triggers, they 
allow you to "unlock" an entire new set of algorithms using just linear combinations of these triggers. This is how speedcubers can learn and retain thousands of algs—they aren't actually learning anything new—they are just learning new ways to combine things that they already understand deeply. It's like how Americans (and a whole lot of other people) learn 26 letters—then these letters unlocks the 
door to speech, reading, etc. In guitar, if you learn a few scales (these scales are our sets of "things"), you can improvise to any song—all you have to do is apply the scale functions to the right key.
    

As an engineer (student), one of the most frequent sets of "things" that I use is applied mathematics. In applied mathematics, you basically start seeing the physical world as combinations of mathematical functions—which are our sets of "things" in this context. This framework of modeling the real world not only allows me to reduce the real world to its essential bits, but also grants me the ability to communicate this information to other people—mathematics provides a lean framework, and also one that can be understood by others.

This had me thinking...what if discovery is just a combinatorics problem given a specific set of "things"? i.e., the enumeration of the space of all combinations of "things". If this is the case, what does it mean to design/create a set of "things"? 

% I wanted to end this blog post by stating how excited I am to have been studying Real Analysis. To try and empathize with my excitement, think about this for a moment. You have dreams of performing at the 
% biggest concerts and stages, but the only thing holding you back is...you can't speak! You have no clue of how to speak or read! But the key is you can \emph{see} yourself performing, soaring, equilibriating through "thing" space. But suddenly a book falls from the sky—
% and its exactly what you need to learn to be able to start flying. That book is Principles of Mathematical Analysis by Walter Rudin. It's not that Real Analysis is my end-all passion for life. Neither is the set of 26 letters of the alphabet to The Weeknd. But this is the mathematical framework of which I can build upon—to enable lifelong, accumulated learning—to be able to start flying.


\end{document}
