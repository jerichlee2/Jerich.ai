\documentclass[12pt]{article}

% Packages
\usepackage[margin=1in]{geometry}
\usepackage{amsmath,amssymb,amsthm}
\usepackage{enumitem}
\usepackage{hyperref}
\usepackage{xcolor}
\usepackage{import}
\usepackage{xifthen}
\usepackage{pdfpages}
\usepackage{transparent}
\usepackage{listings}


\lstset{
    breaklines=true,         % Enable line wrapping
    breakatwhitespace=false, % Wrap lines even if there's no whitespace
    basicstyle=\ttfamily,    % Use monospaced font
    frame=single,            % Add a frame around the code
    columns=fullflexible,    % Better handling of variable-width fonts
}

\newcommand{\incfig}[1]{%
    \def\svgwidth{\columnwidth}
    \import{./Figures/}{#1.pdf_tex}
}
\theoremstyle{definition} % This style uses normal (non-italicized) text
\newtheorem{solution}{Solution}
\newtheorem{proposition}{Proposition}
\newtheorem{problem}{Problem}
\newtheorem{lemma}{Lemma}
\newtheorem{theorem}{Theorem}
\newtheorem{remark}{Remark}
\newtheorem{note}{Note}
\theoremstyle{plain} % Restore the default style for other theorem environments
%

% Theorem-like environments
% Title information
\title{Quotients}
\author{Jerich Lee}
\date{\today}

\begin{document}

\maketitle
Exercises $2.10.1, 2.10.3, 2.10.5, 2.10.6$ \\
Read $\S\S 3.1, 3.2$ for next lecture 
\begin{problem}[2.10.1]
    Let $G$ be the group of invertible real upper triangular $2\times 2$ matrices. Determine whether or not the following conditions describe normal subgroups $H$ of $G$ . If they do, use the First Isomorphism Theorem to identify the quotient group $\frac{G}{H}$
    \noindent
    \begin{enumerate}
        \item $a_{11}=1 $
        \item $a_{12}=0 $
        \item $a_{11} =a_{22} $
        \item $a_{11} =a_{22} =1$    
    \end{enumerate}  
\end{problem}
\begin{solution}
    what is the inverse of $H = \begin{bmatrix}
        1 &  b \\
        0 &  c \\
    \end{bmatrix}$ 
    if $a = \begin{bmatrix}
        a &  b \\
        0 &  c \\
    \end{bmatrix}$, does $a=HaH^{-1}$ 

    Let \( G \) be the group of all invertible upper-triangular \( 2 \times 2 \) real matrices:
\[
G = \left\{
\begin{pmatrix}
a & b \\
0 & c
\end{pmatrix}
\;\middle|\; a, c \in \mathbb{R} \setminus \{0\},\, b \in \mathbb{R}
\right\}.
\]
Let \( A \) be the subgroup of \( G \) consisting of those matrices whose \((1,1)\) entry is \( 1 \):
\[
A = \left\{
\begin{pmatrix}
1 & b \\
0 & c
\end{pmatrix}
\;\middle|\; b \in \mathbb{R},\, c \in \mathbb{R} \setminus \{0\}
\right\}.
\]
We have already established that \( A \trianglelefteq G \) (i.e., \( A \) is a normal subgroup).

\section*{Using the First Isomorphism Theorem}

We want to identify the quotient group \( G / A \). A standard technique is:
\begin{enumerate}
    \item Define a surjective group homomorphism \( \varphi : G \to H \) onto some simpler group \( H \).
    \item Identify the kernel \( \ker(\varphi) \).
    \item Then, by the \textbf{First Isomorphism Theorem}, we have
    \[
    G \big/ \ker(\varphi) \cong \mathrm{im}(\varphi).
    \]
    \item If \( \ker(\varphi) = A \), then \( G / A \cong \mathrm{im}(\varphi) \).
\end{enumerate}

\subsection*{1. A Natural Homomorphism}

Consider the map
\[
\varphi : G \to \mathbb{R}^\times
\quad\text{defined by}\quad
\varphi\left(
\begin{pmatrix}
a & b \\
0 & c
\end{pmatrix}
\right) = a,
\]
where \( \mathbb{R}^\times = \mathbb{R} \setminus \{0\} \) (the multiplicative group of all nonzero reals).

\paragraph{Well-defined:} If \( \begin{pmatrix} a & b \\ 0 & c \end{pmatrix} \in G \), then \( a \neq 0 \), so \( \varphi(g) \in \mathbb{R}^\times \).

\paragraph{Homomorphism:} For any 
\( g_1 = \begin{pmatrix} a_1 & b_1 \\ 0 & c_1 \end{pmatrix} \) and 
\( g_2 = \begin{pmatrix} a_2 & b_2 \\ 0 & c_2 \end{pmatrix} \) in \( G \),
\[
g_1 g_2 = 
\begin{pmatrix}
a_1 a_2 & a_1 b_2 + b_1 c_2 \\
0 & c_1 c_2
\end{pmatrix},
\quad\text{so}\quad
\varphi(g_1 g_2) = a_1 a_2.
\]
On the other hand,
\[
\varphi(g_1) \varphi(g_2) = a_1 \cdot a_2.
\]
Hence, \( \varphi(g_1 g_2) = \varphi(g_1) \varphi(g_2) \).

\paragraph{Surjective:} Given any nonzero real number \( \alpha \), we can map it from
\( \begin{pmatrix} \alpha & 0 \\ 0 & 1 \end{pmatrix} \) (for instance), so 
\( \varphi\left( \begin{pmatrix} \alpha & 0 \\ 0 & 1 \end{pmatrix} \right) = \alpha \).  
Therefore, \( \mathrm{im}(\varphi) = \mathbb{R}^\times \).

\subsection*{2. The Kernel of \( \varphi \)}

\[
\ker(\varphi)
= \left\{
\begin{pmatrix}
a & b \\
0 & c
\end{pmatrix}
\in G
\;\middle|\;
\varphi\left(\begin{pmatrix}
a & b \\
0 & c
\end{pmatrix}\right) = 1
\right\}
= \left\{
\begin{pmatrix}
1 & b \\
0 & c
\end{pmatrix}
: b \in \mathbb{R},\, c \neq 0
\right\}.
\]
But that set is exactly \( A \). Hence, \( \ker(\varphi) = A \).

\subsection*{3. Apply the First Isomorphism Theorem}

By the First Isomorphism Theorem:
\[
G / \ker(\varphi)
\cong
\mathrm{im}(\varphi).
\]
But \( \ker(\varphi) = A \) and \( \mathrm{im}(\varphi) = \mathbb{R}^\times \).  
Therefore,
\[
G / A \cong \mathbb{R}^\times.
\]

\section*{Conclusion}

Using the map \( \varphi \) that “forgets” everything except the top-left entry \( a \), we see that the quotient \( G / A \) is \textbf{isomorphic to} \( \mathbb{R}^\times \) (the multiplicative group of all nonzero reals). Thus,
\[
\boxed{G / A \cong \mathbb{R}^\times.}
\]
What is $\begin{bmatrix}
    a &  b \\
    0 &  c \\
\end{bmatrix}\cdot \begin{bmatrix}
    a &  0 \\
    0 &  b \\
\end{bmatrix}\cdot \begin{bmatrix}
    \frac{1}{a} & 0  \\
     -\frac{b}{ac}&\frac{1}{c}   \\
\end{bmatrix}$ 
\end{solution}
\begin{problem}[2.10.3]
   Let $P$ be a partition of a group $G$ with the property that for any pair of elements $A,B$ of the partition, the product set $AB$ is contained entirely within another element $C$ of the partition. Let $N$ be the element of $P$ which contains $1$ . Prove that $N$ is a normal subgroup of $G$ and that $P$ is the set of its cosets. 
\end{problem}
\begin{solution}
    
\end{solution}
\begin{problem}[2.10.5]
   Identify the quotient group $\frac{\mathbb{{R}}^{\times}}{P}$, where $P$ denotes the subgroup of positive real numbers.  
\end{problem}
\begin{solution}
    
\end{solution}
\begin{problem}[2.10.6]
  Let $H=\left\{ \pm 1, \pm i \right\} $  be the subgroup of $G=\mathbb{{C}}^{\times}$ of fourth roots of unity. Describe the cosets of $H$ in $G$ explicitly, and prove that $\frac{G}{H}$   is isomorphic to $G$. 
\end{problem}
\begin{solution}
    
\end{solution}
\end{document}
