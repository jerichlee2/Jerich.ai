\documentclass[12pt]{article}

% Packages
\usepackage[margin=1in]{geometry}
\usepackage{amsmath,amssymb,amsthm}
\usepackage{enumitem}
\usepackage{hyperref}
\usepackage{xcolor}
\usepackage{import}
\usepackage{xifthen}
\usepackage{pdfpages}
\usepackage{transparent}
\usepackage{listings}


\lstset{
    breaklines=true,         % Enable line wrapping
    breakatwhitespace=false, % Wrap lines even if there's no whitespace
    basicstyle=\ttfamily,    % Use monospaced font
    frame=single,            % Add a frame around the code
    columns=fullflexible,    % Better handling of variable-width fonts
}

\newcommand{\incfig}[1]{%
    \def\svgwidth{\columnwidth}
    \import{./Figures/}{#1.pdf_tex}
}
\theoremstyle{definition} % This style uses normal (non-italicized) text
\newtheorem{solution}{Solution}
\newtheorem*{proposition}{Proposition}
\newtheorem{problem}{Problem}
\newtheorem{lemma}{Lemma}
\theoremstyle{plain} % Restore the default style for other theorem environments
%

% Theorem-like environments
% Title information
\title{Proofs with structure}
\author{Jerich Lee}
\date{\today}

\begin{document}

\maketitle
We now are able to create our own hypotheses—check it out:
\begin{problem}
    Let \(a\) and \(b\) be real numbers and suppose that \(a-5b=4\) and \(b+2=3\). Show that \(a=9\).  
\end{problem}
\begin{solution}
    \begin{proof} Since \(b+2=3\), we have \(b=1\). Therefore:
        \begin{align}
           a&=(a-5b)+5b \\[10pt] 
            &=4+5 \cdot1 \\[10pt] 
            &=9
        \end{align}
    \end{proof}
\end{solution}
Writing this in Lean:
\begin{lstlisting}
example {a b : \real} (h1 : a - 5 * b = 4) (h2 : b + 2 = 3) : a = 9 := by
    have hb : b = 1 := by addarith [h2]
    calc
      a = a - 5 * b + 5 * b := by ring
      _ = 4 + 5 * 1 := by rw [h1, hb]
      _ = 9 := by ring
\end{lstlisting}
Pay attention to the \emph{have} command on line 2 of the above; we can create our own givens and use it in our proofs. Coolio.
\vspace{.5cm} 
We can use lemmas, or previously proved statements to help prove our current statement. Check it out:
\begin{problem}
    Let \(x\) be a rational number, and suppose that \(3x=2\). Show that \(x\neq 1\).  
\end{problem}
\begin{solution}
    \begin{proof} It suffices to prove that \(x<1\): 
        \begin{align}
           x&=\frac{3x}{3} \\[10pt] 
           &= \frac{2}{3}\\[10pt] 
            &<1
        \end{align}
    \end{proof}
\end{solution}
Lean:
\begin{lstlisting}
example {x : \rat} (hx : 3 * x = 2) : x \neq 1 := by
    apply ne_of_lt
    calc
      x = 3 * x / 3 := by ring
      _ = 2 / 3 := by rw [hx]
      _ < 1 := by numbers
\end{lstlisting} 
\end{document}
