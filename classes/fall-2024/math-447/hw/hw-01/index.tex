\documentclass[12pt]{article}

% Packages
\usepackage[margin=1in]{geometry}
\usepackage{amsmath,amssymb,amsthm}
\usepackage{enumitem}
\usepackage{hyperref}
\usepackage{xcolor}
\renewcommand{\labelenumi}{(\alph{enumi})} % Change numbering to (a), (b), ...

% Define the solution environment with normal text
\theoremstyle{definition} % This style uses normal (non-italicized) text
\newtheorem{solution}{Solution}
\newtheorem*{proposition}{Proposition}
\newtheorem{problem}{Problem}
\newtheorem{lemma}{Lemma}
\theoremstyle{plain} % Restore the default style for other theorem environments
%


% Title information
\title{MATH 447: Real Variables - Homework \#10}
\author{Jerich Lee}
\date{\today}

\begin{document}

\maketitle

\begin{problem}[1.8 a]
The principle of mathematical induction can be extended as follows. A list $P_{m}, P_{m+1}, \dots$ of propositions is true provided (i) $P_{m}$ is true, (ii) $P_{n+1}$ is true whenever $P_{n}$ is true and $n\geq m$.
Prove $n^2>n+1$ for all integers $n\geq2$.
\end{problem}

\begin{solution}
\begin{proof}
    We will prove the base case of $n=2$ first:
\begin{align}
(2)^2>(2)+1
\end{align}
\begin{align}
4>3
\end{align}
The above verifies the base case. The inductive hypothesis is as follows:

\begin{align}
n^2>n+1, \ n\geq 2
\end{align}

We want to prove the case such that $P_{n+1}$ is true whenever $P_{n}$ is true and $n\geq m$. The inductive step is as follows:

\begin{align}
(n+1)^2 &> (n+1)+1 \\
n^2+2n+1 &> n+2 \\
n^2+n &> 1 \\
n(n+1) &> 1
\end{align}

By the inductive hypothesis, $n^2>n+1, n\geq 2$,
\begin{align}
n(n^2) &> n(n+1) > 1 \\
n^3 &> 1
\end{align}

The last line is true for all $n\geq 2$.  
\end{proof}

\end{solution}
\begin{problem}[2.8]
    Find all rational solutions of the equation $x^8-4x^5+13x^3-7x+1=0$.
\end{problem}

\begin{solution}
    To find all rational solutions of the equation $x^8-4x^5+13x^3-7x+1=0$, we can use the Rational Zeros Theorem.

Corollary—Rational Zeros Theorem: If a polynomial equation 
\begin{align}
a_nx^n + a_{n-1}x^{n-1} + ... + a_1x + a_0 = 0
\end{align}
with integer coefficients has a rational solution $x = \frac{p}{q}$ 
—where $p$ and $q$ are integers with no common factors and $q \neq 0$—, then:
\begin{enumerate}
    \item $p$ must be a factor of the constant term $a_0$
    \item $q$ must be a factor of the leading coefficient $a_n$
\end{enumerate}

First, let's identify the coefficients:
\begin{enumerate}
    \item $c_8 = 1$
    \item $c_5 = -4$
    \item $c_3 = 13$
    \item $c_1 = -7$
    \item $c_0 = 1$
\end{enumerate}

According to the theorem, if $\frac{c}{d}$ is a rational solution (where $c$ and $d$ are integers with no common factors and $d \neq 0$), then:
\begin{enumerate}
    \item $c$ must divide $c_0 = 1$
    \item $d$ must divide $c_8 = 1$
\end{enumerate}

The only integers that divide 1 are 1 and -1. Therefore, the only possible rational solutions are:
\begin{enumerate}
    \item $\frac{1}{1} = 1$
    \item $\frac{-1}{1} = -1$
\end{enumerate}

Now, we need to check if these candidates actually satisfy the equation:

For $x = 1$:
\begin{align}
1^8 - 4(1^5) + 13(1^3) - 7(1) + 1 = 1 - 4 + 13 - 7 + 1 = 4 \neq 0
\end{align}

For $x = -1$:
\begin{align}
(-1)^8 - 4((-1)^5) + 13((-1)^3) - 7(-1) + 1 = 1 + 4 - 13 + 7 + 1 = 0
\end{align}

Therefore, the only rational solution to the equation $x^8-4x^5+13x^3-7x+1=0$ is $x = -1$.
\end{solution}


\begin{problem}[3.5]
    \begin{enumerate}
        \item Show $|b|\leq a$ if and only if $-a\leq b\leq a$.
        \item Prove $||a|-|b||\leq|a-b|$ for all $a,b \in \mathbb{R}$.
    \end{enumerate}
\end{problem}

\begin{solution}
 \begin{proof}
        \begin{enumerate}
            \item If $|b| \leq a$, then by definition, $-a \leq b \leq a$. This is because $|b| \leq a$ implies $b$ is within the interval $[-a, a]$.
            \item If $-a \leq b \leq a$, then $b$ is within the interval $[-a, a]$. This directly implies $|b| \leq a$ since the maximum deviation of $b$ from zero is $a$.
        \end{enumerate}
        Thus, $|b| \leq a$ if and only if $-a \leq b \leq a$. 
    \end{proof}
    \begin{proof}
        We will prove this inequality using the triangle inequality and considering both possible cases.
  \textbf{Using the Triangle Inequality:}
            \begin{enumerate}
                \item The triangle inequality states $|a| = |(a - b) + b| \leq |a - b| + |b|$.
                \item Rearranging gives $|a| - |b| \leq |a - b|$.
            \end{enumerate}
            
          \textbf{Consider the Reverse Situation:}
            \begin{enumerate}
                \item Similarly, $|b| = |(b - a) + a| \leq |b - a| + |a| = |a - b| + |a|$.
                \item Rearranging gives $|b| - |a| \leq |a - b|$.
            \end{enumerate}
            
             \textbf{Combine the Results:}
            \begin{enumerate}
                \item From the two inequalities, we have:
                \begin{align}
                |a| - |b| \leq |a - b| \quad \text{and} \quad |b| - |a| \leq |a - b|
                \end{align}
            \end{enumerate}
            
            \textbf{Conclusion:}
            \begin{enumerate}
                \item The absolute value $||a| - |b||$ is defined as:
                \begin{align}
                ||a| - |b|| = \max(|a| - |b|, |b| - |a|)
                \end{align}
                \item Therefore, $||a| - |b|| \leq |a - b|$.
            \end{enumerate}

        
        Thus, we have proven that $||a|-|b||\leq|a-b|$ for all $a,b \in \mathbb{R}$. 
    \end{proof}

\end{solution}







\begin{problem}[3.8]
 Let $a,b \in \mathbb{R}$. Show if $a\leq b_{1}$ for every $b_{1} > b$, then $a\leq b$.
   
\end{problem}

\begin{solution}
    \begin{proof}
     We will prove this statement using a proof by contradiction.
         Assume the hypothesis: For every $b_1 > b$, we have $a \leq b_1$.
         Suppose, for the sake of contradiction, that $a > b$.
         Consider $b_1 = \frac{a + b}{2}$. Note that:
        \begin{enumerate}
            \item $b_1 > b$ (because $a > b$)
            \item $b_1 < a$ (because it's the midpoint between $a$ and $b$)
        \end{enumerate}
        By our initial assumption, since $b_1 > b$, we must have $a \leq b_1$.
        However, we also showed that $b_1 < a$.
         This is a contradiction: we can't have both $a \leq b_1$ and $b_1 < a$.

    
    Therefore, our supposition that $a > b$ must be false. We conclude that $a \leq b$.
    
    Thus, we have shown that if $a \leq b_1$ for every $b_1 > b$, then $a \leq b$. 
    \end{proof}
    \end{solution}


\begin{problem}[4.1 r]
    
For each set below that is bounded above, list three upper bounds for the set. Otherwise write "NOT BOUNDED ABOVE" or "NBA".

$\cap_{n=1}^\infty\left( 1-\frac{1}{n}, 1+\frac{1}{n} \right)$ 

\end{problem}
\begin{solution}    
It is observed that the intersection of the sequences above converges to 1 as $n$ approaches infinity. Therefore, 2, 3, and 4 are all upper bounds for the set.
\end{solution}

\begin{problem}[4.8]
 Let $S$ and $T$ be nonempty subsets of $\mathbb{R}$ with the following property:
$s\leq t$ for all $s \in S$ and $t \in T$.

   
\end{problem}
\begin{solution}
         Observe $S$ is bounded above and $T$ is bounded below.
         Prove $\text{sup}\ S \leq \text{inf}\ T$
         Give an example of such sets $S$ and $T$ where $S \ \cap \ T$ is nonempty.
         Give an example of sets $S$ and $T$ where $\text{sup}\ S = \text{inf}\ T$ and $S \ \cap \ T$ is the empty set.

    

        \begin{enumerate}
            \item Let $M=t,\ t \in T$. Then $S\leq M$ for all $s\in S$. By Def 4.2, $M$ is an upper bound of $S$, and $S$ is bounded above.
            \item Let $m=s$, such that $s\in S$. Then, $m\leq t$ for all $t \in T$. Then, $m$ is a lower bound of $T$, and $T$ is bounded below.
        \end{enumerate}
        
     To prove $\sup S \leq \inf T$:
        \begin{enumerate}
            \item By the given property, we know that $s \leq t$ for all $s \in S$ and all $t \in T$.
            \item Let $M = \sup S$. By definition of supremum, $s \leq M$ for all $s \in S$.
            \item For any $t \in T$, we have $s \leq t$ for all $s \in S$.
            \item Therefore, $M = \sup S \leq t$ for all $t \in T$.
            \item Since $M \leq t$ for all $t \in T$, $M$ is a lower bound for $T$.
            \item By definition of infimum, $\inf T$ is the greatest lower bound of $T$.
            \item Thus, $M \leq \inf T$.
        \end{enumerate}
        Therefore, we have proven that $\sup S \leq \inf T$. 
    
        
         An example of such sets $S$ and $T$ where $S \ \cap \ T$ is nonempty: $S=[0,1], T=[1,2]$
        
       An example of sets $S$ and $T$ where $\text{sup}\ S = \text{inf}\ T$ and $S \ \cap \ T$ is the empty set: $S=[0,1), T=(1,2]$
    
\end{solution}

\section*{Problem 7 (8.5)}
\begin{problem}[8.5]
 \begin{enumerate}
    \item Consider three sequences $(a_{n}), (b_n)$ and $(s_{n})$ such that $a_{n}\leq s_{n}\leq b_{n}$ for all $n$ and $\lim_{ n \to \infty }a_{n}=\lim_{ n \to \infty }b_{n}=s$. Prove $\lim_{ n \to \infty }s_{n}=s$. This is called the "squeeze lemma".
\end{enumerate}
  
\end{problem}
\begin{solution}
     \begin{enumerate}
        \item Given: For all $\varepsilon > 0$, there exist $N_1, N_2 \in \mathbb{N}$ such that for $n > N_1$ and $n > N_2$:
        \begin{enumerate}
            \item $|a_n - s| < \varepsilon$, which implies $s - \varepsilon < a_n < s + \varepsilon$
            \item $|b_n - s| < \varepsilon$, which implies $s - \varepsilon < b_n < s + \varepsilon$
        \end{enumerate}
        \item We also know that for all $n \in \mathbb{N}$, $a_n \leq s_n \leq b_n$
        \item Combining these facts, we can conclude that for $n > \max(N_1, N_2)$:
        \begin{align}
        s - \varepsilon < a_n \leq s_n \leq b_n < s + \varepsilon
        \end{align}
        \item This implies:
        \begin{align}
        s - \varepsilon < s_n < s + \varepsilon
        \end{align}
        \item Therefore:
        \begin{align}
        |s_n - s| < \varepsilon
        \end{align}
        \item By the definition of a limit of a sequence, this proves that $\lim_{n \to \infty} s_n = s$
    \end{enumerate}
    Thus, we have proven the squeeze lemma. 

    \begin{enumerate}[resume]
        \item Suppose $(s_{n})$ and $(t_{n})$ are sequences such that $|s_{n}|\leq t_{n}$ for all $n$ and $\lim_{ n \to \infty }t_{n}=0$. Prove $\lim_{ n \to \infty }s_{n}=0$.
    \end{enumerate}
    \begin{enumerate}
        \item Given: $\lim_{n \to \infty} t_n = 0$, so for any $\varepsilon > 0$, there exists an $N \in \mathbb{N}$ such that for all $n > N$:
        \item $|t_n - 0| < \varepsilon$
        \item This implies: $|t_n| < \varepsilon$
        \item We know that $|s_n| \leq t_n$ for all $n$, so: $|s_n| \leq |t_n| < \varepsilon$
        \item This means: $-\varepsilon < s_n < \varepsilon$
        \item Therefore: $|s_n - 0| < \varepsilon$
    \end{enumerate}
    By the definition of a limit, this proves that $\lim_{n \to \infty} s_n = 0$. 
    
     
    \end{solution}


\begin{problem}[8.6]
    Let $(s_{n})$ be a sequence in $\mathbb{R}$.
\begin{enumerate}
    \item Prove $\lim_{ n \to \infty }s_{n}=0$ if and only if $\lim_{ n \to \infty }|s_{n}|=0$
\end{enumerate}

\end{problem}

\begin{solution}
   \begin{proof}
    We will prove this statement in two parts:
\begin{enumerate}
    \item If $\lim_{n \to \infty} s_n = 0$, then $\lim_{n \to \infty} |s_n| = 0$:
    \begin{enumerate}
        \item Given $\lim_{n \to \infty} s_n = 0$, for every $\epsilon > 0$, there exists an $N$ such that for all $n \geq N$, $|s_n - 0| < \epsilon$.
        \item This simplifies to $|s_n| < \epsilon$.
        \item Therefore, $\lim_{n \to \infty} |s_n| = 0$.
    \end{enumerate}
    \item If $\lim_{n \to \infty} |s_n| = 0$, then $\lim_{n \to \infty} s_n = 0$:
    \begin{enumerate}
        \item Given $\lim_{n \to \infty} |s_n| = 0$, for every $\epsilon > 0$, there exists an $N$ such that for all $n \geq N$, $|s_n| < \epsilon$.
        \item This directly implies that $|s_n - 0| < \epsilon$, so $\lim_{n \to \infty} s_n = 0$.
    \end{enumerate}
\end{enumerate}
Thus, we have proven that $\lim_{n \to \infty} s_n = 0$ if and only if $\lim_{n \to \infty} |s_n| = 0$. 

\begin{enumerate}[resume]
    \item Observe that if $s_{n}=(-1)^n$, then $\lim_{ n \to \infty }|s_{n}|$ exists, but $\lim_{ n \to \infty }s_{n}$ does not exist.
\end{enumerate}

\textbf{Observation:} For the sequence $s_n = (-1)^n$, we can see that $s_n$ alternates between -1 and 1.

To prove that $\lim_{n \to \infty} s_n$ does not exist, we can establish two subsequences:
\begin{enumerate}
    \item $s_{n_1}$: the subsequence of even terms, where $s_{n_1} = 1$ for all $n$
    \item $s_{n_2}$: the subsequence of odd terms, where $s_{n_2} = -1$ for all $n$
\end{enumerate}

Clearly, $\lim_{n \to \infty} s_{n_1} = 1$ and $\lim_{n \to \infty} s_{n_2} = -1$

Since these two subsequences converge to different values, we can conclude that $\lim_{n \to \infty} s_n$ does not exist, demonstrating that the sequence is divergent.

However, $\lim_{n \to \infty} |s_n| = 1$ does exist, as $|s_n| = 1$ for all $n$.

   \end{proof} 
\end{solution}

\begin{problem}[Bonus]
    Use the completeness of $\mathbb{R}$ to show the existence of $x>0$ with $x^2=2$. Specifically, consider $S=\{t\in \mathbb{R} : t>0, t^2<2\}$. Clearly, $S$ is nonempty $(1\in S)$. Further, $2$ is an upper bound for $S$. Indeed, suppose $t\in S$, then $(2-t)(2+t)=4-x^2>4-2>0$. Clearly, $2+t>0$, hence $2-t>0$. 
Let $x=\text{sup} \ S$. Prove that $x^2=2$, by establishing that (i) $x^2\leq 2$, and (ii) $x^2\geq 2$. Once these inequalities are established, we can conclude that $x^2=2$.

\end{problem}
\begin{solution}
    \begin{proof}
        \begin{enumerate}
            \item Let $\alpha = \sup S$. We will prove that $\alpha^2 = 2$ by showing that $\alpha^2 \leq 2$ and $\alpha^2 \geq 2$.
            \item First, let's prove $\alpha^2 \leq 2$:
            \begin{enumerate}
                \item Assume $\alpha^2 > 2$.
                \item For $n \in \mathbb{N}$, consider $(\alpha - \frac{1}{n})^2$:
                \begin{align}
                \left( \alpha-\frac{1}{n} \right)^2=\alpha^2-\frac{2\alpha}{n}+\frac{1}{n^2}>\alpha^2-\frac{2\alpha}{n}
                \end{align}
                \item Since $\alpha^2 > 2$, we have:
                \begin{align}
                2<\alpha^2-\frac{2\alpha}{n} \\
                2-\alpha^2<-\frac{2\alpha}{n} \\
                \alpha^2-2> \frac{2\alpha}{n} \\
                \frac{\alpha^2-2}{2\alpha}> \frac{1}{n}
                \end{align}
                \item Choose $n_0 \in \mathbb{N}$ such that $\frac{1}{n_0} < \frac{\alpha^2-2}{2\alpha}$.
                \item Then:
                \begin{align}
                \left( \alpha-\frac{1}{n_0} \right)^2>\alpha^2-(\alpha^2-2)=2
                \end{align}
                \item This contradicts the fact that $\alpha$ is an upper bound for $S$.
                \item Therefore, our assumption must be false, and $\alpha^2 \leq 2$.
            \end{enumerate}
            \item Now, let's prove $\alpha^2 \geq 2$:
            \begin{enumerate}
                \item Assume $\alpha^2 < 2$.
                \item For $n \in \mathbb{N}$, consider $(\alpha + \frac{1}{n})^2$:
                \begin{align}
                \left( \alpha + \frac{1}{n} \right)^2 = \alpha^2 + \frac{2\alpha}{n} + \frac{1}{n^2} < \alpha^2 + \frac{2\alpha}{n} + \frac{1}{n} = \alpha^2 + \frac{2\alpha + 1}{n}.
                \end{align}
                \item Choose $n_0 \in \mathbb{N}$ such that $\frac{1}{n_0} < \frac{2 - \alpha^2}{2\alpha + 1}$.
                \item Then:
                \begin{align}
                \left( \alpha + \frac{1}{n_0} \right)^2 < \alpha^2 + (2 - \alpha^2) = 2.
                \end{align}
                \item This means $\alpha + \frac{1}{n_0} \in S$, contradicting $\alpha$ as an upper bound for $S$.
                \item Therefore, our assumption must be false, and $\alpha^2 \geq 2$.
            \end{enumerate}
            \item Since we have shown $\alpha^2 \leq 2$ and $\alpha^2 \geq 2$, we can conclude that $\alpha^2 = 2$.
        \end{enumerate}
        
        Thus, we have proven the existence of a real number $\alpha > 0$ such that $\alpha^2 = 2$. 
        
    \end{proof}
\end{solution}


\end{document}